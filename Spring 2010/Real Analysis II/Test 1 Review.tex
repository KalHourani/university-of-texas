
\documentclass[12pt,leqno]{article}

\usepackage{graphicx,color,amsmath,amsfonts,amssymb,amscd,amsthm,amsbsy,upref}


\textheight=8.5truein
\textwidth=6.0truein
\hoffset=-.5truein
\voffset=-.5truein
\numberwithin{equation}{section}
\pagestyle{headings}
\footskip=36pt

\newcommand{\question}[2] {\vspace{.25in} \noindent\fbox{#1} #2 \vspace{.10in}}

\swapnumbers
\newtheorem{thm}{Theorem}[section]
\newtheorem{hthm}[thm]{*Theorem}
\newtheorem{lem}[thm]{Lemma}
\newtheorem{cor}[thm]{Corollary}
\newtheorem{prop}[thm]{Proposition}
\newtheorem{con}[thm]{Conjecture}
\newtheorem{exer}[thm]{Exercise}
\newtheorem{bpe}[thm]{Blank Paper Exercise}
\newtheorem{apex}[thm]{Applications Exercise}
\newtheorem{ques}[thm]{Question}
\newtheorem{scho}[thm]{Scholium}
\newtheorem*{Exthm}{Example Theorem}
\newtheorem*{Thm}{Theorem}
\newtheorem*{Con}{Conjecture}
\newtheorem*{Axiom}{Axiom}

\theoremstyle{definition}
\newtheorem*{Ex}{Example}
\newtheorem*{Def}{Definition}
\newtheorem*{Lem}{Lemma}

\newcommand{\lcm}{\operatorname{lcm}}
\newcommand{\ord}{\operatorname{ord}}
\def\pfrac#1#2{{\left(\frac{#1}{#2}\right)}}


\makeindex

\begin{document}


\thispagestyle{plain}
\begin{flushright}

\end{flushright}
{\center\section*{{\huge Mid-Term Review}}}

\begin{Thm}
 Let $Q$ be a closed cube in $\mathbb{R}^n$, and let $C(Q)$ denote the set of continuous real-valued functions on $Q$. Then $C(Q)$ is a vector space under point-wise and scalar multiplication. Further, let \[||f||_{\infty}=\text{sup}_{x\in Q}|f(x)|\] Then $C(Q)$ is a normed vector space under norm $f\cdot g=||f-g||_{\infty}$, and $C(Q)$ is a complete metric space under this norm.
\end{Thm}

\begin{proof}
 
\end{proof}

\begin{Lem}[Cauchy Schwarz]
 \[\left|\int_{\mathbb{R}^n}f(x)f(x)dx\right|\leq\left(\int_{\mathbb{R}^n}|f|^2dx\right)^{1/2}\left(\int_{\mathbb{R}^n}|g|^2dx\right)^{1/2}\]
\end{Lem}

\begin{proof}
 Take $\lambda\in\mathbb{R}$. Then \begin{align*}0&\leq\int_{\mathbb{R}^n}\left|f(x)-\lambda g(x)\right|^2dx=\int_{\mathbb{R}^n}(f(x)-\lambda g(x))(f(x)-\lambda g(x))dx\\&=\int_{\mathbb{R}^n}|f(x)|^2dx-2\lambda\int_{\mathbb{R}^n}f(x)g(x)dx+\lambda^2\int_{\mathbb{R}^n}|g(x)|^2dx\end{align*} Let \begin{align*}A&=\int_{\mathbb{R}^n}|g(x)|^2dx\\B&=\int_{\mathbb{R}^n}f(x)g(x)dx\\C&=\int_{\mathbb{R}^n}|f(x)|^2dx\end{align*} Then $0\leq A\lambda^2+2B\lambda+C$ is a polynomial in $\mathbb{R}$. This is true only if the discriminant, $(-2B)^2-4AC$, is not positive, i.e.\begin{align*}(-2B)^2-4AC&\leq0\\4B^2-4AC&\leq0\\B^2-AC&\leq0\\B^2&\leq AC\\B&\leq\sqrt{AC}\\\left|\int_{\mathbb{R}^n}f(x)f(x)dx\right|&\leq\left(\int_{\mathbb{R}^n}|f|^2dx\right)^{1/2}\left(\int_{\mathbb{R}^n}|g|^2dx\right)^{1/2}\end{align*}
\end{proof}



\end{document}

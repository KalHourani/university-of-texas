
\documentclass[12pt,leqno]{article}

\usepackage{graphicx,color,amsmath,amsfonts,amssymb,amscd,amsthm,amsbsy,upref}


\textheight=8.5truein
\textwidth=6.0truein
\hoffset=-.5truein
\voffset=-.5truein
\numberwithin{equation}{section}
\pagestyle{headings}
\footskip=36pt

\newcommand{\question}[2] {\vspace{.25in} \noindent\fbox{#1} #2 \vspace{.10in}}

\swapnumbers
\theoremstyle{definition}
\newtheorem{thm}{Theorem}[section]
\newtheorem{hthm}[thm]{*Theorem}
\newtheorem{lem}[thm]{Lemma}
\newtheorem{cor}[thm]{Corollary}
\newtheorem{prop}[thm]{Proposition}
\newtheorem{con}[thm]{Conjecture}
\newtheorem{exer}[thm]{Exercise}
\newtheorem{bpe}[thm]{Blank Paper Exercise}
\newtheorem{apex}[thm]{Applications Exercise}
\newtheorem{ques}[thm]{Question}
\newtheorem{scho}[thm]{Scholium}
\newtheorem*{Exthm}{Example Theorem}
\newtheorem*{Thm}{Theorem}
\newtheorem*{Con}{Conjecture}
\newtheorem*{Axiom}{Axiom}

\newtheorem*{Ex}{Example}
\newtheorem*{Def}{Definition}
\newtheorem*{Lem}{Lemma}

\newcommand{\lcm}{\operatorname{lcm}}
\newcommand{\ord}{\operatorname{ord}}
\def\pfrac#1#2{{\left(\frac{#1}{#2}\right)}}


\makeindex

\begin{document}


\thispagestyle{plain}
\begin{flushright}
\large{\textbf{Khalid Hourani\\
Heitmann\\
Homework 7\\
}}
\end{flushright}

\section*{6.6 Decomposition of V: Jordan Form}

\question{1}{If $S$ and $T$ are nilpotent linear transformations which commute, prove that $ST$ and $S+T$ are nilpotent linear transformations.}

\begin{proof}
 Suppose $S^n=T^m=0$. Then $(ST)^n=S^nT^n=0$, so $ST$ is nilpotent. By the binomial theorem, \[(S+T)^{mn}=\sum_{k=1}^{mn}{mn\choose k}S^{mn-k}T^k\] If $S^{mn-k}\not=0$, then $mn-k<n$, hence $k>mn-n=n(m-1)>m$, so $T^k=0$. Thus, the expression above yields $(S+T)^{mn}=0$.
\end{proof}

\question{2}{By a direct matrix computation, show that}\[\begin{pmatrix}0&1&0&0\\0&0&1&0\\0&0&0&0\\0&0&0&0\end{pmatrix}\text{ and }\begin{pmatrix}0&1&0&0\\0&0&1&0\\0&0&0&1\\0&0&0&0\end{pmatrix}\] are not similar.

\begin{proof}
 Suppose $M$ is a matrix so that \[M\begin{bmatrix}0&1&0&0\\0&0&1&0\\0&0&0&0\\0&0&0&0\end{bmatrix}=\begin{bmatrix}0&1&0&0\\0&0&1&0\\0&0&0&1\\0&0&0&0\end{bmatrix}M\]and write \[M=\begin{bmatrix}m_{11}&m_{12}&m_{13}&m_{14}\\m_{21}&m_{22}&m_{23}&m_{24}\\m_{31}&m_{32}&m_{33}&m_{34}\\m_{41}&m_{42}&m_{43}&m_{44}\end{bmatrix}\] Then \begin{align*}\begin{bmatrix}m_{11}&m_{12}&m_{13}&m_{14}\\m_{21}&m_{22}&m_{23}&m_{24}\\m_{31}&m_{32}&m_{33}&m_{34}\\m_{41}&m_{42}&m_{43}&m_{44}\end{bmatrix}\begin{bmatrix}0&1&0&0\\0&0&1&0\\0&0&0&0\\0&0&0&0\end{bmatrix}&=\begin{bmatrix}0&1&0&0\\0&0&1&0\\0&0&0&1\\0&0&0&0\end{bmatrix}\begin{bmatrix}m_{11}&m_{12}&m_{13}&m_{14}\\m_{21}&m_{22}&m_{23}&m_{24}\\m_{31}&m_{32}&m_{33}&m_{34}\\m_{41}&m_{42}&m_{43}&m_{44}\end{bmatrix}\\\begin{bmatrix}0&m_{11}&m_{12}&0\\0&m_{21}&m_{22}&0\\0&m_{31}&m_{32}&0\\0&m_{41}&m_{42}&0\end{bmatrix}&=\begin{bmatrix}m_{21}&m_{22}&m_{23}&m_{24}\\m_{31}&m_{32}&m_{33}&m_{34}\\m_{41}&m_{42}&m_{43}&m_{44}\\0&0&0&0\end{bmatrix}\end{align*}which implies \[M=\begin{bmatrix}m_{11}&m_{12}&m_{13}&m_{14}\\0&m_{11}&m_{12}&0\\0&0&m_{11}&0\\0&0&0&0\end{bmatrix}\] which is certainly not invertible. 
\end{proof}

\question{5}{}
\begin{description}
 \item [(a)] Prove that the matrix \[\begin{pmatrix}1&1&1\\-1&-1&-1\\1&1&0\end{pmatrix}\] is nilpotent, and find its invariants and Jordan form.
 \item [(b)] Prove that the matrix in part \textbf{(a)} is not similar to \[\begin{pmatrix}1&1&1\\-1&-1&-1\\1&0&0\end{pmatrix}\]
\end{description}

\begin{proof}
 \begin{description}
  \item [(a)]  Notice that \[\begin{bmatrix}1&1&1\\-1&-1&-1\\1&1&0\end{bmatrix}^3=0\] Thus, by theorem 6.5.1, its only invariant is 3, and its Jordan form is \[\begin{bmatrix}0&1&0\\0&0&1\\0&0&0\end{bmatrix}\]
  \item [(b)] Notice that this matrix is not nilpotent, so it is not similar to $\begin{bmatrix}1&1&1\\-1&-1&-1\\1&1&0\end{bmatrix}$
 \end{description}
\end{proof}

\question{12}{Find all possible Jordan forms for}
\begin{description}
 \item [(a)] all $8\times8$ matrices having $x^2(x-1)^3$ as minimal polynomial;
 \item [(b)] all $10\times10$ matrices, over a field of characteristic different from 2, having $x^2(x-1)^2(x+1)^3$ as minimal polynomial
\end{description}

\begin{proof}[Solution]\indent
 \begin{description}
  \item [(a)] The matrix's largest 0-block is $2\times2$ and the matrix's largest 1-block is $3\times3$. Thus, the remaining possible block combinations are $1\times1,1\times1,1\times2$; $1\times1,2\times2$; $3\times3$, with every block being either a 0-block or a 1-block, except for the $3\times3$ block, which can only be a 1-block.
 \end{description}
  \item [(b)] The matrix's largest 0-block is $2\times2$, the matrix's largest 1-block is $2\times2$, and the matrix's largest -1-block is $3\times3$.
\end{proof}

\question{14}{If $T$ is in $A(V)$ then $T$ is diagonalizable (if all its characteristic roots are in $F$) if and only if whenever $v(T-\lambda)^m=0$, for $v\in V$ and $\lambda\in F$, then $v(T-\lambda)=0$.}

\question{15}{Using the result of Problem 14, prove that if $E^2=E$ then $E$ is diagonalizable.}

\question{19}{If $A,B\in F$ are diagonalizable and if they commute, prove that there is an element $C\in F_n$ such that both $CAC^{-1}$ and $CBC^{-1}$ are diagonal.}

\end{document}

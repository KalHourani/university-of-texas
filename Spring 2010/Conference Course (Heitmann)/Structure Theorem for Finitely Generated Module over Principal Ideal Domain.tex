\documentclass[12pt,leqno]{article}

\usepackage{graphicx,color,amsmath,amsfonts,amssymb,amscd,amsthm,amsbsy,upref}


\textheight=8.5truein
\textwidth=6.0truein
\hoffset=-.5truein
\voffset=-.5truein
\numberwithin{equation}{section}
\pagestyle{empty}
\footskip=36pt

\newcommand{\question}[2] {\vspace{.25in} \fbox{#1} #2 \vspace{.10in}}
\renewcommand{\part}[1] {\vspace{.10in} {\bf (#1)}}

\swapnumbers
\newtheorem{thm}{Theorem}[section]
\newtheorem{hthm}[thm]{*Theorem}
\newtheorem{lem}[thm]{Lemma}
\newtheorem{cor}[thm]{Corollary}
\newtheorem{prop}[thm]{Proposition}
\newtheorem{con}[thm]{Conjecture}
\newtheorem{exer}[thm]{Exercise}
\newtheorem{bpe}[thm]{Blank Paper Exercise}
\newtheorem{apex}[thm]{Applications Exercise}
\newtheorem{ques}[thm]{Question}
\newtheorem{scho}[thm]{Scholium}
\newtheorem*{Exthm}{Example Theorem}
\newtheorem*{Thm}{Theorem}
\newtheorem*{Con}{Conjecture}
\newtheorem*{Axiom}{Axiom}

\theoremstyle{definition}
\newtheorem*{Ex}{Example}
\newtheorem*{Def}{Definition}
\newtheorem*{Lem}{Lemma}
\newtheorem*{Cor}{Corollary}

\newcommand{\lcm}{\operatorname{lcm}}
\newcommand{\ord}{\operatorname{ord}}
\newcommand{\Tor}{\operatorname{Tor}}
\def\pfrac#1#2{{\left(\frac{#1}{#2}\right)}}


\makeindex

\begin{document}

\thispagestyle{plain}
\begin{flushright}
\large{\textbf{Khalid Hourani\\
}}
\end{flushright}
\section{Finitely Generated Modules over Principal Ideal Domains}
\begin{Def}
 Let $R$ be a ring and $M$ an $R$-Module. We say $M$ is \textit{Noetherian} if, whenever \[M_1\subseteq M_2\subseteq M_3\subseteq\hdots\] is a chain of submodules, there is a $k\in\mathbb{N}$ with $M_n=M_k$ for all $n\geq k$. We say $R$ is \textit{Noetherian} if it is Noetherian as a module over itself.
\end{Def}

\begin{thm}
 Let $R$ be a ring and $M$ an $R$-module. The following are equivalent
\begin{description}
 \item [(1)] $M$ is a Noetherian $R$-module.
 \item [(2)] Every non-empty set of submodules of $M$ contains a maximal element under inclusion.
 \item [(3)] Every submodule of $M$ is finitely generated.
\end{description}
\end{thm}

\begin{proof}
 $[\textbf{(1)}\rightarrow\textbf{(2)}]$ We shall prove the contrapositive. Suppose $\Sigma$ is a nonempty set of submodules of $M$ with no maximal element under inclusion. If $M_1\in\Sigma$, then $M_1$ is not maximal, so there must exist an $M_2\in\Sigma$ with $M_1\subseteq M_2$. Similarly, $M_2$ is not maximal, so there must exist an $M_3$ such that $M_2\subseteq M_3$, and so on. Hence, there exists an infinitely increasing chain of submodules, \[M_1\subseteq M_2\subseteq M_3\subseteq\hdots\] so $M$ is not Noetherian.

$[\textbf{(2)}\rightarrow\textbf{(3)}]$ Suppose $N$ is a submodule of $M$ and let $\Sigma$ be the collection of all finitely generated submodules of $N$. Since $\{0\}$ is a finitely generated submodule, it is in $\Sigma$. Thus, $\Sigma$ is non-empty. Then $\Sigma$ contains a maximal element, say $N'$. In order to reach a contradiction, suppose $N'\not=N$. Since $N'\in\Sigma$, it is generated by a finite number of elements, say $\{a_1,a_2,\hdots,a_n\}$. Now, let $x\in N-N'$. Then $N'$ is a proper subset of the module generated by $\{a_1,a_2,\hdots,a_n,x\}$, which contradicts the maximality of $N'$. Thus, $N'=N$. So $N\in\Sigma$ is a finitely generated submodule.

$[\textbf{(3)}\rightarrow\textbf{(1)}]$ Suppose \[M_1\subseteq M_2\subseteq M_3\subseteq\hdots\] is a chain of submodules of $M$. Let \[N=\bigcup_{i=1}^{\infty}M_i\] and observe that $N$ is a submodule of $M$. Thus, $N$ is finitely generated, say by $\{a_1,a_2,\hdots,a_n\}$. Then each $a_i$ is in some $M_{j_i}$. Let $m=\text{max}\{j_1,j_2,\hdots,j_n\}$. Then each $a_i$ is in $M_m$, so $N=(a_1,a_2,\hdots,a_n)$ is a subset of $M_m$. Further, $M_m\subseteq N$, so $N=M_k$ for all $k\geq m$.
\end{proof}

\begin{cor}
 If $R$ is a Principal Ideal Domain, then every nonempty set of ideals of $R$ has a maximal element and $R$ is a Noetherian ring.
\end{cor}

\begin{proof}
 Since $R$ is a Principal Ideal Domain, every ideal in $R$ is finitely generated. That is, it satisfies \textbf{(3)} of the above theorem with $M=R$.
\end{proof}

\begin{Def}
Let $R$ be a ring and $M$ an $R$-module. $M$ is said to be \textit{free}, or \textit{free of rank $n$} on the subset $A$ of $M$ if for every nonzero element $x$ of $M$, there exist unique nonzero elements $r_1,r_2,\hdots,r_n$ of $R$ and unique $a_1,a_2,\hdots,a_n$ in $A$ such that $x=r_1a_1+r_2a_2+\hdots+r_na_n$, for some $n\in\mathbb{Z}^+$. In this situation, we say $A$ is a \textit{basis} or \textit{set of free generators} for $M$.
\end{Def}

\begin{prop}
 Let $R$ be an integral domain and let $M$ be a free $R$-module of rank $n<\infty$. Then any $n+1$ elements of $M$ are $R$-linearly dependent, that is, for any $y_1,y_2,\hdots,y_n,y_{n+1}\in M$, there are elements $r_1,r_2,\hdots,r_{n+1}\in R$, not all $0$, such that \[r_1y_1+r_2y_2+\hdots+r_ny_n+r_{n+1}y_{n+1}=0\]
\end{prop}

\begin{proof}
Let $F$ be the quotient field of $R$. Since \[M\cong\underbrace{R\oplus R\oplus\hdots\oplus R}_{n-\text{times}}\] we have that \[M\subseteq\underbrace{F\oplus F\oplus F\hdots\oplus F}_{n-\text{times}}\] The latter is an $n$ dimensional vector space over $F$, so any $n+1$ elements are linearly dependent over $F$. That is, if $y_1,y_2,\hdots,y_n,y_{n+1}$ are in $M$, then there exist $\alpha_1,\alpha_2,\hdots,\alpha_{n+1}\in F$ not all $0$ so that \[\alpha_1y_1+\alpha_2y_2+\hdots+\alpha_{n+1}y_{n+1}=0\] Since $F$ is the quotient field of $R$, we have that $\alpha_i=\frac{a_i}{b_i}$ for some $a_i,b_i\in R$. In other words, \[\frac{a_1}{b_1}y_1+\frac{a_2}{b_2}y_2+\hdots+\frac{a_{n+1}}{b_{n+1}}y_{n+1}=0\] Let $\beta_i=\frac{a_ib_1b_2\hdots b_n}{b_i}$. Observe that $\beta_i$ is in $R$, and that not all $\beta_i$ are $0$. Now, multiplying by $b_1b_2\hdots b_n$, we see that \[\beta_1y_1+\beta_2y_2+\hdots+\beta_{n+1}y_{n+1}=0\] Thus, any $n+1$ elements of $M$ are $R$-linearly dependent.
\end{proof}

\begin{Def}
 For any integral domain $R$ the \textit{rank} of an $R$-module $M$ is the maximum number of $R$-linearly independent elements of $M$.
\end{Def}

Notice that this definition agrees with the definition of rank given for free modules.

\begin{thm}
Let $R$ be a Principal Ideal Domain, let $M$ be a free $R$-module of finite rank $n$ and let $N$ be a submodule of $M$. Then
\begin{description}
 \item [(1)] $N$ is free of rank $m$, $m\leq n$ and
 \item [(2)] there exists a basis $y_1,y_2,\hdots,y_n$ of $M$ so that $a_1y_1,a_2y_2,\hdots,a_my_m$ is a basis of $N$ where $a_1,a_2,\hdots,a_m$ are nonzero elements of $R$ with divisibility relations \[a_1|a_2|\hdots|a_m\]
\end{description}
\end{thm}

\begin{proof}
 If $N=\{0\}$, we are done. Hence, suppose $N\not=\{0\}$. If $\phi:M\to R$ is a homomorphism, then $\phi(N)$ is an ideal in $R$. Since $R$ is a Principal Ideal Domain, $\phi(N)$ is generated by a single element, say $a_{\phi}$ for some $a_{\phi}\in R$. Let \[\Sigma=\{(a_{\phi})|\phi\in\text{Hom}(M,R)\}\] To see that $\Sigma$ is nonempty, observe that $\{0\}\in\Sigma$ by taking $\phi\equiv0$. By part \textbf{(2)} of \textbf{Theorem 1.1}, $\Sigma$ has a maximal element. That is, there is a homomorphism $\psi:M\to R$ such that the ideal $\psi(N)=(a_{\psi})$ is maximal. In particular, it is not contained in any other element of $\Sigma$. Let $a_1=a_{\psi}$ and choose $y\in N$ such that $\psi(y)=a_1$.

To see that $a_1$ is not nonzero, let $\{x_1,x_2,\hdots,x_n\}$ be a basis of $M$ and let $\pi_i:M\to R$ be given by $\pi_i(r_1x_1+r_2x_2+\hdots+r_nx_n)=r_i$. Since $N\not=\{0\}$, $\pi_i(N)\not=0$ for some $i$. Thus, there is a non-trivial ideal in $\Sigma$. Since $(a_1)$ is maximal, $a_1\not=0$.

Let $\phi\in\text{Hom}(M,R)$ and let $d$ be a generator of the ideal generated by $a_1$ and $\phi(y)$. Thus, $d$ is a divisor of both $a_1$ and $\phi(y)$. Then $d=r_1a_1+r_2\phi(y)$ for some $r_1,r_2\in R$. Consider the homomorphism $\Lambda:M\to R$ given by $\Lambda=r_1\psi+r_2\phi$. Then $\Lambda(y)=(r_1\psi+r_2\phi)(y)=r_1a_1+r_2\phi(y)=d$. Thus, $d\in\Lambda(N)$, so $(d)\in\Lambda(N)$. But $d|a_1$ so $(a_1)\subseteq(d)\subseteq\Lambda(N)$ and, since $(a_1)$ is maximal, $(a_1)=(d)$. Thus, $a_1|\phi(y)$ since $d|\phi(y)$.

Applying this to the projection homomorphism $\pi_i$, we see that $a_1|\pi_i(y)$ for all $i$. Let $\pi_i(y)=a_1b_i$ for some $b_i\in R$. Now, let \[y_1=\sum_{i=1}^nb_ix_i\] and observe that \begin{align*}ay_1&=a_1\sum_{i=1}^nb_ix_i\\&=\sum_{i=1}^n\pi_i(y)x_i\\&=y\end{align*}Then $a_1=\psi(y)=\psi(a_1y_1)=a_1\psi(y_1)$. Since $a_1\not=0$, $\psi(y_1)=1$.

Now, let $x\in M$. Then \begin{align*}\psi(x-\psi(x)y_1)&=\psi(x)-\psi(x)\psi(y_1)\\&=\psi(x)-\psi(x)\cdot1\\&=0\end{align*} so $x-\psi(x)y_1$ is in the kernel of $\psi$. Writing $x=\psi(x)y_1+(x-\psi(x)y_1)$ shows that $x$ can be written as the sum of an element in $Ry_1$ and an element in $\ker\psi$. Further, if $ry_1\in\ker\psi
$, then $0=\psi(ry_1)=r\psi(y_1)=r$ shows that \[M=Ry_1\oplus\ker\psi\] Hence, $y_1$ can be taken as an element for a basis of $M$.

Consider $x\in N$. Observe that $\psi(x)$ is divisible by $a_1$ since $a_1$ generates $\psi(N)$. Let $\psi(x)=ba_1$. Then \begin{align*}\psi(x-ba_1y_1)&=\psi(x)-ba_1\psi(y_1)\\&=ba_1-ba_1\cdot1\\&=0\end{align*} so $x-ba_1y_1$ is in the kernel of $\psi$. Further, $x-ba_1y_1\in N$, so $x-ba_1y_1\in N\cap\ker\psi$ Writing $x=ba_1y_1+(x-ba_1y_1)$ shows that $x$ can be written as the sum of an element in $Ra_1y_1$ and $N\cap\ker\psi$. If $ra_1y_1\in N\cap\ker\psi$, then $0=\psi(ra_1y_1)=ra_1\psi(y_1)=ra_1$. Since $a_1\not=0$ and $R$ is an integral domain, $r=0$. Thus, \[N=Ra_1y_1\oplus(N\cap\ker\psi)\] Hence, $a_1y_a$ can be taken as  an element for a basis of $N$.

In order to prove \textbf{(1)}, we proceed by induction on the rank, $m$, of $N$. If $m=0$, then $N=0$, since $M$ is a free $R$-module. Thus, $N$ is free. Suppose that, if $N$ has rank $m-1$, then $N$ is free. Now suppose $N$ has rank $m$. Since $N=Ra_1y_1\oplus(N\cap\ker\psi)$, $N\cap\ker\psi$ has rank $m-1$, and so it is a free $R$-module of rank $m-1$. If $\beta$ is a basis of $N\cap\ker\psi$, then $\beta\cup\{a_1y_1\}$ is a basis of $N$, so $N$ is free of rank $m$.

To prove \textbf{(2)}, we proceed by induction on the rank, $n$, of $M$. If $n=0$, the theorem is trivially true. Suppose that, if $M$ has rank $n-1$, then there exists a basis $y_1,y_2,\hdots,y_{n-1}$ of $M$ so that $a_1y_1,a_2y_2,\hdots,a_my_m$ is a basis of $N$ where $a_1,a_2,\hdots,a_m$ are nonzero elements of $R$ with divisibility relations \[a_1|a_2|\hdots|a_m\] Now, suppose $M$ has rank $n$. By \textbf{(1)}, $\ker\psi$ is a free submodule. Further, since $M=Ry_1\oplus\ker\psi$, $\ker\psi$ has rank $n-1$. Applying \textbf{(1)} to $\ker\psi$ and its submodule $N\cap\ker\psi$, there is a basis $y_2,y_3,\hdots,y_n$ of $\ker\psi$ such that $a_2y_2,a_3y_3,\hdots,a_my_m$ is a basis of $N\cap\ker\psi$ with $a_2|a_3|\hdots|a_m$. Since $M=Ry_1\oplus\ker\psi$ and $  N=Ra_1y_1\oplus(N\cap\ker\psi)$, $y_1,y_2,\hdots,y_n$ is a basis of $M$ and $a_1y_1,a_2y_2,\hdots,a_my_m$ is a basis of $N$. To see that $a_1|a_2$, let $\phi:M\to R$ be given by $\phi(y_1)=\phi(y_2)=1$ and $\phi(y_i)=0$ for $i\geq2$. Then $\phi(a_1y_1)=a_1\phi(y_1)=a_1$, so $a_1\in\phi(N)$. Moreover, $(a_1)\subseteq\phi(N)$. Since $(a_1)$ is maximal in $\Sigma$, $(a_1)=\phi(N)$. Further, $\phi(a_2y_2)=a_2\phi(y_2)=a_2$, so $a_2\in\phi(N)$. Hence, $a_2\in(a_1)$, so $a_1|a_2$.
\end{proof}

\begin{Def}
 Let $R$ be an integral domain and $M$ an $R$-module. Then \[\text{Tor}(M)=\{x\in M|rx=0\text{ for some nonzero }r\in R\}\] If $\text{Tor}(M)=0$, $M$ is said to be \textit{Torsion Free}. If $\text{Tor}(M)=M$, then $M$ is a \textit{Torsion Module}.
\end{Def}

\begin{thm}
 Let $R$ be a Principal Ideal Domain and let $M$ be a finitely generated $R$-module. Then
\begin{description}
 \item [(1)] $M$ is isomorphic to the direct sum of finitely many cyclic modules. More precisely \[M\cong R^r\oplus R/(a_1)\oplus R/(a_2)\oplus\hdots\oplus R/(a_m)\] for some integer $r\geq0$ and nonzero elements $a_1,a_2,\hdots,a_m$ of $R$ which are not units in $R$ and which satisfy the divisibility relations \[a_1|a_2|\hdots|a_m\]
 \item [(2)] $M$ is torsion free if and only if $M$ is free.
 \item [(3)] In the decomposition in \textbf{(1)}, \[\Tor(M)\cong R/(a_1)\oplus R/(a_2)\oplus\hdots\oplus R/(a_m)\] In Particular, $M$ is a torsion module if and only if $r=0$ and, in this case, the annihilator of $M$ is the ideal $(a_m)$.
\end{description}
\end{thm}

\begin{proof}
 Since $M$ is a finitely generated $R$-module, there is a set $\{x_1,x_2,\hdots,x_n\}$ of minimal cardinality which generates $M$. Let $R^n$ be the free module of rank $n$ with basis $b_1,b_2,\hdots,b_n$ and let $\pi:R^n\to M$ be given by $\pi(b_i)=x_i$. Observe that $\pi$ is surjective since $x_1,x_2,\hdots,x_n$ generate $M$. By \textbf{The First Isomorphism Theorem for Modules}, $R^n/\ker\pi\cong M$. By \textbf{Theorem 4}, we can choose another basis $y_1,y_2,\hdots,y_n$ of $R^n$ so that $a_1y_1,a_2y_2,\hdots,a_my_m$ is a basis of $\ker\pi$ for some $a_1,a_2,\hdots,a_m$ in $R$ satisfying $a_1|a_2|\hdots|a_m$. Then \[M\cong R^n/\ker\pi=(Ry_1\oplus Ry_2\oplus\hdots\oplus Ry_n)/(Ra_1y_1\oplus Ra_2y_2\oplus\hdots\oplus Ra_my_m)\] Let \[\phi:Ry_1\oplus Ry_2\oplus\hdots\oplus Ry_n\to R/(a_1)\oplus R/(a_1)\oplus\hdots\oplus R/(a_m)\oplus R^{n-m}\] be given by \[\phi(\alpha_1y_1,\alpha_2y_2,\hdots,\alpha_ny_n)=(\alpha_1\bmod{(a_1)},\alpha_2\bmod{(a_2)},\hdots,\alpha_m\bmod{(a_m)},\alpha_{m+1},hdots,\alpha_n)\] Observe that the kernel of $\phi$ is the set of all elements where $a_i|\alpha_i$ for all $i$. Hence, \[\ker\phi=Ra_1y_1\oplus Ra_2y_2\oplus\hdots\oplus Ra_my_m\] Applying \textbf{The First Isomorphism Theorem for Modules} again yields \[M\cong R/(a_1)\oplus R/(a_2)\oplus\hdots\oplus R/(a_m)\oplus R^{n-m}\] If $a$ is a unit in $R$, then $R/(a)=0$. Thus, by removing any $a_i$ which are units and setting $r=m-n$, we have \textbf{(1)}.

If $a\not=0$, then $\Tor(R/(a))=R/(a)$. In other words, $R/(a)$ is a torsion module. Thus, \textbf{(1)} implies that $M$ is torsion free if and only if $M\cong R^r$, and so we have \textbf{(2)}. Further, the annihilator of $R/(a)$ is the ideal $(a)$, and the annihilator of $R^{n-m}$ is $(0)$, so \[\Tor(M)\cong R/(a_1)\oplus R/(a_2)\oplus\hdots\oplus R/(a_m)\] If $M$ is a torsion module, then \[M=\Tor(M)\cong R/(a_1)\oplus R/(a_2)\oplus\hdots\oplus R/(a_m)\] and, by \textbf{(1)}, $r=0$. In this case, since $a_1|a_2|\hdots|a_m$, the annihilator of $M$ is $(a_m)$.
\end{proof}
\end{document}


\documentclass[12pt,leqno]{article}

\usepackage{graphicx,color,amsmath,amsfonts,amssymb,amscd,amsthm,amsbsy,upref}


\textheight=8.5truein
\textwidth=6.0truein
\hoffset=-.5truein
\voffset=-.5truein
\numberwithin{equation}{section}
\pagestyle{headings}
\footskip=36pt

\newcommand{\question}[2] {\vspace{.25in} \noindent\fbox{#1} #2 \vspace{.10in}}

\swapnumbers
\newtheorem{thm}{Theorem}[section]
\newtheorem{hthm}[thm]{*Theorem}
\newtheorem{lem}[thm]{Lemma}
\newtheorem{cor}[thm]{Corollary}
\newtheorem{prop}[thm]{Proposition}
\newtheorem{con}[thm]{Conjecture}
\newtheorem{exer}[thm]{Exercise}
\newtheorem{bpe}[thm]{Blank Paper Exercise}
\newtheorem{apex}[thm]{Applications Exercise}
\newtheorem{ques}[thm]{Question}
\newtheorem{scho}[thm]{Scholium}
\newtheorem*{Exthm}{Example Theorem}
\newtheorem*{Thm}{Theorem}
\newtheorem*{Con}{Conjecture}
\newtheorem*{Axiom}{Axiom}

\theoremstyle{definition}
\newtheorem*{Ex}{Example}
\newtheorem*{Def}{Definition}
\newtheorem*{Lem}{Lemma}

\newcommand{\lcm}{\operatorname{lcm}}
\newcommand{\ord}{\operatorname{ord}}
\def\pfrac#1#2{{\left(\frac{#1}{#2}\right)}}


\makeindex

\begin{document}


\thispagestyle{plain}
\begin{flushright}
\large{\textbf{Khalid Hourani\\
}}
\end{flushright}

\question{4}{ }
\begin{description}
 \item [(a)] Let $\mathbb{R}$ be the field of real numbers and $\mathbb{Q}$ the field of rational numbers. In $\mathbb{R}$, $\sqrt{2}$ and $\sqrt{3}$ are both algebraic over $\mathbb{Q}$. Exhibit a polynomial of degree $4$ satisfied by $\sqrt{2}+\sqrt{3}$.
 \item [(b)] What is the degree of $\sqrt{2}+\sqrt{3}$ over $\mathbb{Q}$? Prove your answer.
 \item [(c)] What is the degree of $\sqrt{2}\sqrt{3}$ over $\mathbb{Q}$?
\end{description}

\begin{proof}[Solution]
 \begin{description}
  \item [(a)] $x^4-10x^2+1$
  \item [(b)] Since the degrees of $\sqrt{2}$ and $\sqrt{3}$ are $2$, the degree of $\sqrt{2}+\sqrt{3}$ is at most $4$. Let $x=\sqrt{2}+\sqrt{3}$. To see that $x$ has degree $4$, observe that \begin{align*}x^0&=1\\x^1&
=\sqrt{2}+\sqrt{3}\\x^2&=5+2\sqrt{6}\end{align*} which implies that $x$ has degree at most $4$. So $x$ has degree $4$.
  \item [(c)] Let $x=\sqrt{2}\sqrt{3}=\sqrt{6}$. Observe that \begin{align*}x^0&=1\\x^1&=\sqrt{6}\end{align*} which implies that $x$ has degree of at least $2$. Further, since $x^2-6$ has root $\sqrt{6}$, $x$ has degree of at most $2$. So $x$ has degree $2$.
 \end{description}

\end{proof}

\question{6}{ }
\begin{description}
 \item [(a)] Find an element $u\in\mathbb{R}$ such that $\mathbb{Q}(\sqrt{2},\sqrt[3]{5})=\mathbb{Q}(u)$
 \item [(b)] In $\mathbb{Q}(\sqrt{2},\sqrt[3]{5})$ characterize all the elements $w$ such that $\mathbb{Q}(w)\not=\mathbb{Q}(\sqrt{2},\sqrt[3]{5})$.
\end{description}

\begin{proof}[Solution]
 \begin{description}
  \item [(a)]
  \item [(b)]
 \end{description}

\end{proof}


\question{9}{Suppose that $F$ is a field having a finite number of elements, $q$.}
\begin{description}
 \item [(a)] Prove that there is a prime number $p$ such that $\underbrace{a+a+\hdots+a}_{p-\text{times}}=0$ for all $a\in F$.
 \item [(b)] Prove that $q=p^n$ for some integer $n$.
 \item [(c)] If $a\in F$ prove that $a^q=a$.
 \item [(d)] If $b\in K$ is algebraic over $F$ prove that $b^{q^m}=b$ for some $m>0$.
\end{description}

\begin{proof}
 \begin{description}
  \item [(a)] Since $F$ is finite, it must have positive characteristic, for otherwise the group generated by $1$ would be infinite. Call this characteristic $p$. Observe that $\mathbb{Z}/p\mathbb{Z}$ is isomorphic to the order $p$ subfield $\mathbb{F}_p$ of $F$. Thus, $p$ must be prime, for $F$ has no $0$-divisors. Hence, for all $a\in F$, $\underbrace{a+a+\hdots+a}_{p-\text{times}}=0$
  \item [(b)] The field $F$ is a finite extension of $\mathbb{F}_p$, and therefore a finite dimensional vector space over $\mathbb{F}_p$. Thus, it is isomorphic to $\mathbb{F}_p^n$ for some $n$, and $|F|=q=p^n$.
  \item [(c)] This follows from the definition of a field, for $F^{\times}$ is a group under multiplication. By Lagrange's theorem, $a^{q-1}=1$. Thus, $a^q=a$.
  \item [(d)]
 \end{description}

\end{proof}


\question{14}{ }
\begin{description}
 \item [(a)] Prove that the sum of two algebraic integers is an algebraic integer.
 \item [(b)] Prove that the product of two algebraic integers is an algebraic integer.
\end{description}

\question{15}{ }
\begin{description}
 \item [(a)] Prove that $\sin1^{\circ}$ is an algebraic number.
 \item [(b)] From part \textbf{(a)} prove that $\sin m^{\circ}$ is an algebraic integer for any integer $m$.
\end{description}

\begin{proof}
 \begin{description}
  \item [(a)] $\sin30^{\circ}=\frac{1}{2}$ is clearly algebraic. Moreover, $\sin30^{\circ}$ can be written as an integral polynomial in $\sin1^{\circ}$. Thus, $\sin1^{\circ}$ is algebraic.
  \item [(b)] We observe that \[\sin m^{\circ}=\sin\left(\underbrace{1^{\circ}+1^{\circ}+\hdots+1^{\circ}}_{m-\text{times}}\right)\] Thus, $\sin m^{\circ}$ can be written as an integral polynomial in $\sin1^{\circ}$, so it is algebraic.
 \end{description}

\end{proof}


\question{1}{Using the infinite series for $e$, $e=1+\frac{1}{1!}+\frac{1}{2!}+\frac{1}{3!}+\hdots+\frac{1}{m!}+\hdots$, prove that $e$ is irrational.}

\begin{proof}
 In order to reach a contradiction, suppose that $e=\frac{m}{n}$ for some integers $m$ and $n$. Then \[\frac{m}{n}=1+\frac{1}{1!}+\frac{1}{2!}+\frac{1}{3!}+\hdots\] multiplying by $n!$ \begin{align*}\frac{m}{n}n!&=n!+\frac{n!}{1!}+\frac{n!}{2!}+\frac{n!}{3!}+\hdots+\frac{n!}{n!}+\frac{n!}{(n+1)!}+\hdots\\m(n-1)!&=n!+n!+\frac{n!}{2!}+\frac{n!}{3!}+\hdots+1+\frac{1}{n+1}+\frac{1}{(n+1)(n+2)}+\hdots\end{align*} Let \[k=m(n-1)!-\left(n!+n!+\frac{n!}{2!}+\frac{n!}{3!}+\hdots+1\right)\] Observe that $k$ is an integer, and \begin{align*}k&=\frac{1}{n+1}+\frac{1}{(n+1)(n+2)}+\hdots\\&<\frac{1}{n+1}+\frac{1}{(n+1)^2}+\frac{1}{(n+1)^3}+\hdots\\&=\frac{1}{n}\end{align*} Thus, $k<\frac{1}{n}$. This is a contradiction. Thus, $e$ is irrational.
\end{proof}


\question{4}{If $m>0$ and $n$ are integers, prove that $e^{m/n}$ is transcendental.}

\begin{proof}
 In order to reach a contradiction, suppose $e^{m/n}$ is algebraic of degree $k$. Then $[\mathbb{Z}(e^{m/n}):\mathbb{Z}]=k$.
\end{proof}

\end{document}

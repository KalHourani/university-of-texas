\documentclass[12pt,leqno]{article}

\usepackage{fancyhdr,graphicx,color,amsmath,amsfonts,amssymb,amscd,amsthm,amsbsy,upref}

\title{Algebra II\\\large Homework 8}
\date{March 23, 2011}
\author{Khalid Hourani}

\headheight=14.5pt
\textheight=8.5truein
\textwidth=6.0truein
\hoffset=-.5truein
\voffset=-.5truein
\pagestyle{plain}
\lhead[ ]{ }\lfoot[\large\textbf{\thepage}]{\footnotesize\rightmark}
\chead[ ]{ }\cfoot[ ]{ }
\rhead[ ]{ }\rfoot[\footnotesize\leftmark]{\large\textbf{\thepage}}
\footskip=36pt

\newcommand{\question}[2] {\vspace{.25in}\noindent\fbox{#1} #2 \vspace{.10in}}
\renewcommand{\part}[1] {\vspace{.10in} {\bf (#1)}}
\renewcommand{\headrulewidth}{0.0pt}
\renewcommand{\footrulewidth}{0.04pt}
\theoremstyle{definition}
\newtheorem{thm}{Theorem}
\newtheorem{hthm}[thm]{*Theorem}
\newtheorem{lem}[thm]{Lemma}
\newtheorem{cor}[thm]{Corollary}
\newtheorem{prop}[thm]{Proposition}
\newtheorem{con}[thm]{Conjecture}
\newtheorem{exer}[thm]{Exercise}
\newtheorem{bpe}[thm]{Blank Paper Exercise}
\newtheorem{apex}[thm]{Applications Exercise}
\newtheorem{ques}[thm]{Question}
\newtheorem{scho}[thm]{Scholium}
\newtheorem*{Exthm}{Example Theorem}
\newtheorem*{Thm}{Theorem}
\newtheorem*{Con}{Conjecture}
\newtheorem*{Axiom}{Axiom}

\newtheorem*{Ex}{Example}
\newtheorem*{Def}{Definition}
\newtheorem*{Lem}{Lemma}

\newcommand{\lcm}{\operatorname{lcm}}
\newcommand{\ord}{\operatorname{ord}}
\newcommand{\Gal}{\operatorname{Gal}}
\newcommand{\Aut}{\operatorname{Aut}}
\newcommand{\w}{\omega}
\newcommand{\Z}{\mathbb{Z}}
\newcommand{\Q}{\mathbb{Q}}
\newcommand{\N}{\mathbb{N}}
\newcommand{\R}{\mathbb{R}}
\newcommand{\C}{\mathbb{C}}
\newcommand{\F}{\mathbb{F}}
\newcommand{\+}{\oplus}
\newcommand{\Part}{\center\textbf}
\renewcommand{\labelenumi}{\textbf{\arabic{enumi}.}}
\renewcommand{\labelenumii}{\textbf{(\alph{enumii})}}
\newenvironment{Proof}{\begin{proof}[\textnormal{\textbf{Proof}}]}{\end{proof}}
\newenvironment{Solution}{\begin{proof}[\textnormal{\textbf{Solution}}]}{\end{proof}}
\def\pfrac#1#2{{\left(\frac{#1}{#2}\right)}}
\begin{document}
 \begin{titlepage}
  \maketitle
 \end{titlepage}

\section*{Section 14.7}
  \begin{enumerate}
   \item [4.] 
   \item [5.]
   \item [6.]
   \item [8.] Let $p$, $q$ and $r$ be primes in $\Z$ with $q\not=r$. Let $\sqrt[p]{q}$ denote any root of $x^p-q$ and let $\sqrt[p]{r}$ denote any root of $x^p-r$. Prove that $\Q(\sqrt[p]{q})\not=\Q(\sqrt[p]{r})$.
   \item [9.] \textit{(Artin-Schreir Extensions)} Let $F$ be a field of characteristic $p$ and let $K$ be a cyclic extension of $F$ of degree $p$. Prove that $K=F(\alpha)$ where $\alpha$ is a root of the polynomial $x^p-x+a$ for some $a\in F$. Note that since $F$ contains the $p^{\text{th}}$ roots of unity (namely, 1) that this completes the description of all cyclic extensions of prime degree $p$ over fields containing the $p^{\text{th}}$ roots of unity in all characteristics. 
   \item [12.] Let $L$ be the Galois closure of the finite extension $\Q(\alpha)$ of $\Q$. For any prime $p$ dividing the order of $\Gal(L/\Q)$ prove that there is a subfield $F$ of $L$ with $[L:F]=p$ and $L=F(\alpha)$.
   \begin{Proof}
   Let $H$ be an isomorphic copy of $\Z/p\Z$ inside $\Gal(L/\Q)$ and let $F$ be the fixed field of $H$. By The Fundamental Theorem of Galois Theory, $[L:F]=p$. 
   \end{Proof}
   \item [13.] Let $F$ be a subfield of the real numbers $\R$. Let $a$ be an element of $F$ and let $K=F(\sqrt[n]{a})$ where $\sqrt[n]{a}$ denotes a real $n^{\text{th}}$ root of $a$. Prove that if $L$ is any Galois extension of $F$ contained in $K$ then $[L:F]\leq2$.
  \end{enumerate}
\end{document}

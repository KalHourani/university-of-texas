\documentclass[12pt,leqno]{article}

\usepackage{fancyhdr,graphicx,color,amsmath,amsfonts,amssymb,amscd,amsthm,amsbsy,upref}

\title{Algebra II\\\large Homework 11}
\date{April 11, 2011}
\author{Khalid Hourani}

\headheight=14.5pt
\textheight=8.5truein
\textwidth=6.0truein
\hoffset=-.5truein
\voffset=-.5truein
\pagestyle{plain}
\lhead[ ]{ }\lfoot[\large\textbf{\thepage}]{\footnotesize\rightmark}
\chead[ ]{ }\cfoot[ ]{ }
\rhead[ ]{ }\rfoot[\footnotesize\leftmark]{\large\textbf{\thepage}}
\footskip=36pt

\newcommand{\question}[2] {\vspace{.25in}\noindent\fbox{#1} #2 \vspace{.10in}}
\renewcommand{\part}[1] {\vspace{.10in} {\bf (#1)}}
\renewcommand{\headrulewidth}{0.0pt}
\renewcommand{\footrulewidth}{0.04pt}
\theoremstyle{definition}
\newtheorem{thm}{Theorem}
\newtheorem{hthm}[thm]{*Theorem}
\newtheorem{lem}[thm]{Lemma}
\newtheorem{cor}[thm]{Corollary}
\newtheorem{prop}[thm]{Proposition}
\newtheorem{con}[thm]{Conjecture}
\newtheorem{exer}[thm]{Exercise}
\newtheorem{bpe}[thm]{Blank Paper Exercise}
\newtheorem{apex}[thm]{Applications Exercise}
\newtheorem{ques}[thm]{Question}
\newtheorem{scho}[thm]{Scholium}
\newtheorem*{Exthm}{Example Theorem}
\newtheorem*{Thm}{Theorem}
\newtheorem*{Con}{Conjecture}
\newtheorem*{Axiom}{Axiom}

\newtheorem*{Ex}{Example}
\newtheorem*{Def}{Definition}
\newtheorem*{Lem}{Lemma}

\newcommand{\lcm}{\operatorname{lcm}}
\newcommand{\ord}{\operatorname{ord}}
\newcommand{\Gal}{\operatorname{Gal}}
\newcommand{\Aut}{\operatorname{Aut}}
\newcommand{\w}{\omega}
\newcommand{\Z}{\mathbb{Z}}
\newcommand{\Q}{\mathbb{Q}}
\newcommand{\N}{\mathbb{N}}
\newcommand{\R}{\mathbb{R}}
\newcommand{\C}{\mathbb{C}}
\newcommand{\F}{\mathbb{F}}
\newcommand{\+}{\oplus}
\newcommand{\Part}{\center\textbf}
\renewcommand{\labelenumi}{\textbf{\arabic{enumi}.}}
\renewcommand{\labelenumii}{\textbf{(\alph{enumii})}}
\newenvironment{Proof}{\begin{proof}[\textnormal{\textbf{Proof}}]}{\end{proof}}
\newenvironment{Solution}{\begin{proof}[\textnormal{\textbf{Solution}}]}{\end{proof}}
\def\pfrac#1#2{{\left(\frac{#1}{#2}\right)}}
\begin{document}
 \begin{titlepage}
  \maketitle
 \end{titlepage}

\section*{Section 14.9}
  \begin{enumerate}
   \item [1.] Prove that every purely inseparable extension is normal.
   \begin{Proof}
   Suppose $K/F$ is purely inseparable. Then, for all $\alpha\in K$, $m_{\alpha,F}(x)$ has only one distinct root. This forces $m_{\alpha,K}(x)=(x-\alpha)^n$ for some $n$. In fact, $K$ is the splitting field of the family of such polynomials, hence $K$ is normal.
   \end{Proof}
   \item [2.] 
   \item [3.]
   \item [6.] Show that if $t$ is transcendental over $\Q$ then $\Q(t,\sqrt{t^3-t})$ is not a purely transcendental extension of $\Q$.
   	\begin{Proof}
   	Observe that $\Q(t,\sqrt{t^3-t})$ has transcendence degree 1, for $\{t\}$ is algebraically independent, however $\sqrt{t^3-t}^2-t^3+t=0$, so $\{t,\sqrt{t^3-t}\}$ is not. 
   	\end{Proof}
   \item [7.] Let $k$ be the field with 4 elements, $t$ a transcendental over $k$, $F=k(t^4+t)$ and $K=k(t)$.
   	\begin{enumerate}
			\item Show that $[K:F]=4$. 
			\item Show that $K$ is separable over $F$.
			\item Show that $K$ is Galois over $F$.
			\item Describe the lattice of subgroups of the Galois group and the corresponding lattice of subfields of $K$, giving each subfield in the form $k(r)$, for some rational function $r$.
\end{enumerate}
	\begin{Solution}\indent
	\begin{enumerate}
	\item Let $k=\F_2[\alpha]/(\alpha^2+\alpha+1)$ and set $f(x)=x^4+x+t^4+t\in F[x]$. Clearly, $f(t)=0$, hence $m_{t,F}(x)|f(x)$. However, over $K$ the polynomial $f(x)$ factors as \[f(x)=(x-t)(x-(t+1))(x-(t+\alpha))(x-(t+1+\alpha))\] It is easily verified that the only monic polynomial $g$ which satisfies $(x-t)|g(x)$ and $g(x)|f(x)$ is $x^4+x+t^4+t$, hence $f(x)$ is the minimal polynomial for $t$ over $F$. Thus, $[K:F]=\text{deg}m_{t,F}=4$.
	\item In fact, since the polynomial $f(x)=x^4+x+t^4+t\in F[x]$ has splitting field $F(t,t+1,t+\alpha,t+\alpha+1)=K$, $K$ must be separable over $F$.
	\item Similarly, $K$ is Galois over $F$.
	\item 
\end{enumerate}
	\end{Solution}
  \end{enumerate}
\end{document}

\documentclass[12pt,leqno]{article}

\usepackage{fancyhdr,graphicx,color,amsmath,amsfonts,amssymb,amscd,amsthm,amsbsy,upref}

\title{Algebra II\\\large Homework 4}
\date{February 14, 2011}
\author{Khalid Hourani}

\headheight=14.5pt
\textheight=8.5truein
\textwidth=6.0truein
\hoffset=-.5truein
\voffset=-.5truein
\pagestyle{plain}
\lhead[ ]{ }\lfoot[\large\textbf{\thepage}]{\footnotesize\rightmark}
\chead[ ]{ }\cfoot[ ]{ }
\rhead[ ]{ }\rfoot[\footnotesize\leftmark]{\large\textbf{\thepage}}
\footskip=36pt

\newcommand{\question}[2] {\vspace{.25in}\noindent\fbox{#1} #2 \vspace{.10in}}
\renewcommand{\part}[1] {\vspace{.10in} {\bf (#1)}}
\renewcommand{\headrulewidth}{0.0pt}
\renewcommand{\footrulewidth}{0.04pt}
\theoremstyle{definition}
\newtheorem{thm}{Theorem}
\newtheorem{hthm}[thm]{*Theorem}
\newtheorem{lem}[thm]{Lemma}
\newtheorem{cor}[thm]{Corollary}
\newtheorem{prop}[thm]{Proposition}
\newtheorem{con}[thm]{Conjecture}
\newtheorem{exer}[thm]{Exercise}
\newtheorem{bpe}[thm]{Blank Paper Exercise}
\newtheorem{apex}[thm]{Applications Exercise}
\newtheorem{ques}[thm]{Question}
\newtheorem{scho}[thm]{Scholium}
\newtheorem*{Exthm}{Example Theorem}
\newtheorem*{Thm}{Theorem}
\newtheorem*{Con}{Conjecture}
\newtheorem*{Axiom}{Axiom}

\newtheorem*{Ex}{Example}
\newtheorem*{Def}{Definition}
\newtheorem*{Lem}{Lemma}

\newcommand{\lcm}{\operatorname{lcm}}
\newcommand{\ord}{\operatorname{ord}}
\newcommand{\Gal}{\operatorname{Gal}}
\newcommand{\Aut}{\operatorname{Aut}}
\newcommand{\Z}{\mathbb{Z}}
\newcommand{\Q}{\mathbb{Q}}
\newcommand{\N}{\mathbb{N}}
\newcommand{\R}{\mathbb{R}}
\newcommand{\C}{\mathbb{C}}
\newcommand{\F}{\mathbb{F}}
\newcommand{\Part}{\center\textbf}
\renewcommand{\labelenumi}{\textbf{\arabic{enumi}.}}
\renewcommand{\labelenumii}{\textbf{(\alph{enumii})}}
\newenvironment{Proof}{\begin{proof}[\textnormal{\textbf{Proof}}]}{\end{proof}}
\newenvironment{Solution}{\begin{proof}[\textnormal{\textbf{Solution}}]}{\end{proof}}
\def\pfrac#1#2{{\left(\frac{#1}{#2}\right)}}
\begin{document}
 \begin{titlepage}
  \maketitle
 \end{titlepage}

\section*{Section 13.6}
  \begin{enumerate}
   \item [1.] Suppose $m$ and $n$ are relatively prime positive integers. Let $\zeta_m$ be a primitive $m^{\text{th}}$ root of unity and let $\zeta_n$ be a primitive $n^{\text{th}}$ root of unity. Prove that $\zeta_m\zeta_n$ is a primitive $mn^{\text{th}}$ root of unity.
    \begin{Proof}
    Clearly, $(\zeta_m\zeta_n)^{mn}=1$. Suppose, by way of contradiction, that $(\zeta_m\zeta_n)^k=1$ for some $k<mn$. Then $\zeta_m^k=\zeta_n^{-k}$, hence $\langle\zeta_m^k\rangle=\langle\zeta_n^{-k}\rangle$, from which we see that $|\langle\zeta_m^k\rangle|$ divides $m$. However, $|\langle\zeta_n^{-k}\rangle|$ divides $n$, so $|\langle\zeta_m^k\rangle|$ is a common divisor of $m$ and $n$. This is a contradiction, hence the order of $\zeta_m\zeta_n$ must be $mn$.
    \end{Proof}
   \item [4.] Prove that if $n=p^km$ where $p$ is a prime and $m$ is relatively prime to $p$ then there are precisely $m$ distinct $n^{\text{th}}$ roots of unity over a field of characteristic $p$.
    \begin{Proof}
     Write \[x^n-1=x^{p^km-1}-1=(x^m-1)^{pk}\] Now, since $(m,p)=1$, $x^m-1$ is irreducible and separable, hence $x^m-1$ has exactly $m$ distinct roots. In particular, this means that $x^n-1$ has exactly $m$ distinct roots.
    \end{Proof}
   \item [15.] Let $p$ be an odd prime not dividing $m$ and let $\varPhi_m(x)$ be the $m^{\text{th}}$ cyclotomic polynomial. Suppose $a\in\Z$ satisfies $\varPhi_m(a)\equiv0\pmod{p}$. Prove that $a$ is relatively prime to $p$ and that the order of $a$ in $(\Z/p\Z)^{\times}$ is precisely $m$.
    \begin{Proof}
     Write \[x^m-1=\varPhi_m(x)\prod_{d|m}\varPhi_d(x)\] If $a\in\Z$ satisfies $\varPhi_m(a)\equiv0\pmod{p}$, then $a^m\equiv1\pmod{p}$, hence $(a,p)=1$. Moreover, if the order of $a$, say $d$, is less than $m$, then $a^d\equiv1\pmod{p}$, hence $\varPhi_d(a)\equiv0\bmod{p}$, which would imply that $x^m-a$ has $a$ as a multiple root, a contradiction. In particular, $a$ has order $m$. 
    \end{Proof}
   \item [16.] Let $a\in\Z$. Show that if $p$ is an odd prime dividing $\varPhi_m(a)$ then either $p$ divides $m$ or $p\equiv1\pmod{m}$.
    \begin{Proof}
     By the previous exercise, $a$ has order $m$ in $(\Z/p\Z)^{\times}$, hence $m|p-1$ by Lagrange's Theorem. Equivalently, $p-1\equiv0\pmod{m}$, hence $p\equiv1\pmod{p}$.
    \end{Proof}
  \end{enumerate}


\section*{Section 14.1}
\begin{enumerate}
   \item [1.] 
    \begin{enumerate}
     \item Show that if the field $K$ is generated over $F$ by the elements $\alpha_1,\hdots,\alpha_n$ then an automorphism $\sigma$ of $K$ fixing $F$ is uniqely determined by $\sigma(\alpha_1),\hdots,\sigma(\alpha_n)$. In particular, show that an automorphism fixes $K$ if and only if it fixes a set of generators for $K$.
     \item Let $G\leq\Gal(K/F)$ be a subgroup of the Galois group of the extension $K/F$ and suppose $\sigma_1,\hdots,\sigma_k$ are generators for $G$. Show that the subfield $E/F$ is fixed by $G$ if and only if it is fixed by the generators $\sigma_1,\hdots,\sigma_k$.
    \end{enumerate}
    \begin{Proof}\indent
     \begin{enumerate}
      \item Since $K$ is generated over $F$ by $\alpha_1,\hdots,\alpha_n$, we can write every element $k\in K-F$ as a polynomial in $n$ variables over $F$ evaluated at $\alpha_1,\hdots,\alpha_n$, i.e. we can write $k=P(\alpha_1,\hdots,\alpha_n)$ for some polynomial $P(x_1,\hdots,x_n)\in F[x_1,\hdots,x_n]$. Then $\sigma(k)=\sigma(P(\alpha_1,\hdots,\alpha_n))=P(\sigma(\alpha_1),\hdots,\sigma(\alpha_n))$, hence $\sigma$ is determined by its values on $\alpha_1,\hdots,\alpha_n$.
      \item If $E/F$ is fixed by $G$, then the generators clearly fix $E/F$. Conversely, if the generators fix $E/F$, then, for any $\sigma\in G$, we can write $\sigma=\sigma_{i_1}\sigma_{i_2}\hdots\sigma_{i_j}$ for some $i_1,i_2,\hdots, i_j$. Hence, for any $x\in E$, \[\sigma(x)=\sigma_{i_1}\sigma_{i_2}\hdots\sigma_{i_j}(x)=x\] therefore $G$ fixes $E/F$.\qedhere
     \end{enumerate}
    \end{Proof}
   \item [4.] Prove that $\Q(\sqrt{2})$ and $\Q(\sqrt{3})$ are not isomorphic. 
    \begin{Proof}
     If $\phi:\Q(\sqrt{2})\to\Q(\sqrt{3})$ is an isomorphism, then \[2=\phi(2)=\phi(\sqrt{2}^2)=\phi(\sqrt{2})^2\] However no element $x$ of $\Q(\sqrt{3})$ satisfies $x^2=2$, so there is no isomorphism from $\Q(\sqrt{2})$ to $\Q(\sqrt{3})$. 
    \end{Proof}
   \item [5.] Determine the automorphisms of the extension $\Q(\sqrt[4]{2})/\Q(\sqrt{2})$ explicitly.
    \begin{Proof}
     Suppose $\phi$ is an automorphism of $\Q(\sqrt[4]{2})$ which fixes the elements of $\Q(\sqrt{2})$. We see that $\phi$ is determined by its value on $\sqrt[4]{2}$ and that $\phi(\sqrt{2})=\sqrt{2}$. However, $\phi(\sqrt[4]{2})^2=\phi(\sqrt[4]{2}^2)=\phi(\sqrt{2})=\sqrt{2}$ hence $\phi(\sqrt[4]{2})=\pm\sqrt[4]{2}$, i.e., there are two automorphisms of $\Q(\sqrt[4]{2})/\Q(\sqrt{2})$, determined by: \begin{align*}\sqrt[4]{2}&\mapsto\sqrt[4]{2}\\\sqrt[4]{2}&\mapsto-\sqrt[4]{2}\end{align*} These maps are in fact just the maps \begin{align*}\phi(x)&=x\\\phi(a+b\sqrt[4]{2})&=a-b\sqrt[4]{2}\qedhere\end{align*} 
    \end{Proof}
   \item [10.] Let $K$ be an extension of the field $F$. Let $\phi:K\to K'$ be an isomorphism of $K$ with a field $K'$ which maps $F$ to the subfield $F'$ of $K'$. Prove that the map $\sigma\mapsto\phi\sigma\phi^{-1}$ defines a group isomorphism $\Aut(K/F)\stackrel{\sim}{\to}\Aut(K'/F')$. 
    \begin{Proof}
     We see that the kernel of this map is the function $\text{Id}_K$, for if $\phi\sigma\phi^{-1}=\text{Id}_{K'}$, then $\phi\sigma=\phi$, hence $\sigma=\text{Id}_K$. This map is clearly a homomorphism, for if $\sigma_1,\sigma_2$ are automorphisms of $K/F$, then \[\phi\sigma_1\sigma_2\phi^{-1}=\phi\sigma_1\phi^{-1}\phi\sigma_2\phi^{-1}\] Surjectivity is clear, for if $\tau$ is an automorphism of $K'/F'$, then \[\phi^{-1}\tau\phi\mapsto\tau\] hence the map above describes a group isomorphism. 
    \end{Proof}
\end{enumerate}
\end{document}

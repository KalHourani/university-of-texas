\documentclass[12pt,leqno]{article}

\usepackage{fancyhdr,graphicx,color,amsmath,amsfonts,amssymb,amscd,amsthm,amsbsy,upref}

\title{Algebra II\\\large Homework 7}
\date{March 7, 2011}
\author{Khalid Hourani}

\headheight=14.5pt
\textheight=8.5truein
\textwidth=6.0truein
\hoffset=-.5truein
\voffset=-.5truein
\pagestyle{plain}
\lhead[ ]{ }\lfoot[\large\textbf{\thepage}]{\footnotesize\rightmark}
\chead[ ]{ }\cfoot[ ]{ }
\rhead[ ]{ }\rfoot[\footnotesize\leftmark]{\large\textbf{\thepage}}
\footskip=36pt

\newcommand{\question}[2] {\vspace{.25in}\noindent\fbox{#1} #2 \vspace{.10in}}
\renewcommand{\part}[1] {\vspace{.10in} {\bf (#1)}}
\renewcommand{\headrulewidth}{0.0pt}
\renewcommand{\footrulewidth}{0.04pt}
\theoremstyle{definition}
\newtheorem{thm}{Theorem}
\newtheorem{hthm}[thm]{*Theorem}
\newtheorem{lem}[thm]{Lemma}
\newtheorem{cor}[thm]{Corollary}
\newtheorem{prop}[thm]{Proposition}
\newtheorem{con}[thm]{Conjecture}
\newtheorem{exer}[thm]{Exercise}
\newtheorem{bpe}[thm]{Blank Paper Exercise}
\newtheorem{apex}[thm]{Applications Exercise}
\newtheorem{ques}[thm]{Question}
\newtheorem{scho}[thm]{Scholium}
\newtheorem*{Exthm}{Example Theorem}
\newtheorem*{Thm}{Theorem}
\newtheorem*{Con}{Conjecture}
\newtheorem*{Axiom}{Axiom}

\newtheorem*{Ex}{Example}
\newtheorem*{Def}{Definition}
\newtheorem*{Lem}{Lemma}

\newcommand{\lcm}{\operatorname{lcm}}
\newcommand{\ord}{\operatorname{ord}}
\newcommand{\Gal}{\operatorname{Gal}}
\newcommand{\Aut}{\operatorname{Aut}}
\newcommand{\w}{\omega}
\newcommand{\Z}{\mathbb{Z}}
\newcommand{\Q}{\mathbb{Q}}
\newcommand{\N}{\mathbb{N}}
\newcommand{\R}{\mathbb{R}}
\newcommand{\C}{\mathbb{C}}
\newcommand{\F}{\mathbb{F}}
\newcommand{\+}{\oplus}
\newcommand{\Part}{\center\textbf}
\renewcommand{\labelenumi}{\textbf{\arabic{enumi}.}}
\renewcommand{\labelenumii}{\textbf{(\alph{enumii})}}
\newenvironment{Proof}{\begin{proof}[\textnormal{\textbf{Proof}}]}{\end{proof}}
\newenvironment{Solution}{\begin{proof}[\textnormal{\textbf{Solution}}]}{\end{proof}}
\def\pfrac#1#2{{\left(\frac{#1}{#2}\right)}}
\begin{document}
 \begin{titlepage}
  \maketitle
 \end{titlepage}

\section*{Section 14.3}
  \begin{enumerate}
   \item [1.] Factor $x^8-x$ into irreducibles in $\Z[x]$ and in $\F_2[x]$.
    Write $x^8-x=x(x^7-1)$ and factor $x^7-1$ as \[(x-1)(x^6+x^5+x^4+x^3+x^2+x+1)\] The polynomial $x^6+x^5+x^4+x^3+x^2+x+1$ is the $7^{\text{th}}$ cyclotomic polynomial, which is irreducible. Thus, in $\Z[x]$, our factorization is \[x^8-x=x(x-1)(x^6+x^5+x^4+x^3+x^2+x+1)\] In $\F_2[x]$, this factors as \[x^8-x=x(x-1)(x^3+x^2+1)(x^3+x^2+1)\] That the polynomials $x^3+x^2+1$ and $x^3+x+1$ are irreducible over $\F_2$ is clear: if they were not irreducible, they would have a linear factor, but they clearly have no roots in $\F_2$.
   \item [3.] Prove that an algebraically closed field must be infinite.
    \begin{Proof}
     Suppose $F$ is a field. If $F$ is finite, we can write $F=\F_{p^n}$ for some prime $p$ (recall that a field of characteristic zero must be infinite). However, the polynomial $x^{p^n}-\alpha$ has no roots for $\alpha$ neither 0 nor 1 in $\F_{p^n}$, hence $\F_{p^n}$ is not algebraically closed. Thus, if $F$ is algebraically closed then $F$ is infinite. A similar proof is to show that \[\overline{\F_{p^n}}=\bigcup_{n=1}^{\infty}\F_{p^n}\] from which the result follows. 

     One can also look at the cylcotomic polynomials $\Phi_{p}(x)$ for $p$-prime in $\Z$. The roots of this polynomial are all distinct and differ for each prime $p$, hence there must be infinitely many different roots of unity. In particular, an algebraically closed field, which must contain all roots of unity, must be infinite.  
    \end{Proof}
   \item [4.] Construct the finite field of 16 elements and find a generator for the multiplicative group. How many generators are there?
    \begin{Solution}
     It is easily shown that $x^4+x+1$ is irreducible over $\F_2$, hence \[\F_{16}=\F_2[x]/(x^4+x+1)\] Clearly, $x^{15}=1$ since $\F_{16}^{\times}$ has 15 elements. Moreover, if $x^3=1$ then $x^3-1=0$, which contradicts the fact that $x^4+x+1$ is irreducible. Similarly, if $x^5=1$, then $x^5-1=x(x+1)-1=x^2+x-1=0$, which is also a contradiction. In particular, $x$ has order 15, hence $\F_{16}=\langle x\rangle$. From this, we conclude that the generators for this group are simply \[x,x^2,x^4,x^7,x^8,x^{11},x^{13},x^{14}\] hence there are 8 generators for this group.
    \end{Solution}
   \item [5.] Exhibit an explicit isomorphism between the splitting fields of $x^3-x+1$ and $x^3-x-1$.
    \begin{Solution}
     Rather, we shall show that these splitting fields are in fact \textit{equal}, hence the identity map provides a natural isomorphism. We first see that, if $\alpha$ is a root of $x^3-x+1$, then $\alpha+1$ is a root, since \[(\alpha+1)^3-(\alpha+1)+1=\alpha^3+1-\alpha-1+1=\alpha^3-\alpha+1=0\] Hence the roots of $x^3-x+1$ are $\alpha$, $\alpha+1$ and $\alpha+2$. Moreover, we see that \begin{align*}(2\alpha)^3-(2\alpha)-1=2\alpha^3-2\alpha-1&=2(\alpha^3-\alpha)-1\\&=2(-1)-1\\&=0\end{align*} hence $2\alpha$ is a root of $x^3-x-1$. Similarly, \begin{align*}(2\alpha+1)^3-(2\alpha+1)-1&=(2\alpha)^3+1-(2\alpha)-1-1\\&=(2\alpha)^3-(2\alpha)-1\\&=0\end{align*} hence $2\alpha+1$ and $2\alpha+2$ are also roots of $x^3-x-1$. In particular, the splitting fields of $x^3-x+1$ and $x^3-x-1$ are $\F_3(\alpha)$ and $\F_3(2\alpha)$, respectively, which are clearly equal.
    \end{Solution}
   \item [7.] Prove that 2, 3 or 6 is square in $\F_p$ for every prime $p$. Conclude that the polynomial \[x^6-11x^4+36x^2-36=(x^2-2)(x^3-3)(x^2-6)\] has root modulo $p$ for every prime $p$ but has no root in $\Z$.
    \begin{Proof}
     Take $x$ not 1 in $\F_p^{\times}$ and write $\F_p^{\times}=\langle x\rangle$. Let \[A=\{n^2:n\in\F_p\}\] We shall show that $a\in A$ if and only if $a=x^{2i}$ for some $i$. On the one hand, if $a=x^{2i}$, then $\left(x^i\right)^2=x^{2i}=a$, hence $a$ is a perfect square. Conversely, if $a\in A$, then $a=n^2$ for some $n\in\F_p$. Setting $n=x^i$, we have $a=\left(x^i\right)^2=x^{2i}$. 

We shall now prove that, for any $u,v$ in $\F_p$, at least one of $u$, $v$ or $uv$ must be a perfect square. Suppose that neither $u$ nor $v$ is a perfect square: then $u=x^i$ and $v=x^j$ for some $i,j$ odd. However this forces $uv=x^{i+j}$ with $i+j$ even, hence $uv\in A$. 

In particular, one of 2, 3 and $2\cdot3=6$ must be a perfect square. Clearly then at least one of the polynomials $x^2-2$, $x^2-3$ or $x^2-6$ must have a root in $\F_p$, hence \[(x^2-2)(x^2-3)(x^2-6)=x^6-11x^4+36x^2-36\] has a root modulo $p$ for every prime $p$, however this polynomial clearly has no roots in $\Z$.
    \end{Proof}
   \item [10.] Prove that $n$ divides $\varphi(p^n-1)$. 
    \begin{Proof}
     Let $G=\Z/(p^n-1)\Z$. We see that $\Aut(G)$ has order $\varphi(p^n-1)$, hence it suffices to find an element of order $n$ in $\Aut(G)$, from which we deduce that $n$ divides $\varphi(p^n-1)$ by Lagrange's Theorem. Let $f:G\to G$ be given by \[f(x)=px\] This map is clearly a homomorphism. To see that it is an automorphism we examine its kernel: if $f(x)=px=0$, then \[px\equiv0\pmod{p^n-1}\] hence \[x\equiv0\pmod{p^n-1}\] In particular, its kernel must be trivial, so $f$ is an automorphism. Further, $f^n(x)=p^nx=x$, hence $f^n=\text{Id}_G$. Moreover, if $f^k=\text{Id}_G$, then $f^k(1)=p^k=1$, hence $k=n$. Thus, $f$ has order $n$, and we are done.
    \end{Proof}
   \item [11.] Prove that $x^{p^n}-x+1$ is irreducible over $\F_p$ only when $n=1$ or $n=p=2$. 
    \begin{Proof}
     Suppose that this polynomial is irreducible. We see that, if $\alpha$ is a root of $x^{p^n}-x+1$, then $\alpha+a$ is a root for all $a\in\F_p$, since \begin{align*}(\alpha+a)^{p^n}-(\alpha+a)+1&=\alpha^{p^n}+a-\alpha-a+1\\&=\alpha^{p^n}-\alpha+1\\&=0\end{align*} hence $\F_{p^n}\subseteq\F_p(\alpha)$. This polynomial factors over $\F_p(\alpha)$ as \[\prod_{a\in\F_{p^n}}x-(\alpha+a)\] Now, since $\Gal(\F_p(\alpha)/\F_p)=\langle\sigma\rangle$ for some $\sigma$, we have $\Gal(\F_p(\alpha)/\F_{p^n})=\langle\sigma^n\rangle$. In particular, the automorphism $\sigma^n$ maps $\alpha$ to $\alpha^{p^n}=\alpha-1$, hence the polynomial \[\prod_{a\in\F_p}x-(\alpha+a)\] is fixed by $\sigma^n$, from which we see that it is in $\F_{p^n}[x]$. Thus, it is the minimal polynomial for $\alpha$ over $\F_{p^n}$, and $[\F_p(\alpha):\F_{p^n}]=p$. In particular, we have \begin{align*}[\F_p(\alpha):\F_p]&=[\F_p(\alpha):\F_{p^n}][\F_{p^n}:\F_p]\\p^n&=pn\end{align*} hence either $n=1$ or $n=p=2$. 
    \end{Proof}
  \end{enumerate}

\section*{Section 14.4}
  \begin{enumerate}
   \item [1.] Determine the Galois closure of the field $\Q\left(\sqrt{1+\sqrt{2}}\right)$ over $\Q$
    \begin{Solution}
     Let $\F=\Q\left(\sqrt{1+\sqrt{2}}\right)$. We see that $\F(i)$ is Galois over $\Q$.
    \end{Solution}
   \item [2.] Find a primitive generator for $\Q\left(\sqrt{2},\sqrt{3},\sqrt{5}\right)$ over $\Q$.
    \begin{Solution}
     If $\sigma$ is any automorphism of $\Q\left(\sqrt{2},\sqrt{3},\sqrt{5}\right)$, then $\sigma$ maps $\sqrt{2},\sqrt{3},\sqrt{5}$ to $\pm\sqrt{2},\pm\sqrt{3},\pm\sqrt{5}$, respectively, hence no nontrivial subgroup of the Galois group of $\Q\left(\sqrt{2},\sqrt{3},\sqrt{5}\right)$ fixes $\Q\left(\sqrt{2}+\sqrt{3}+\sqrt{5}\right)$. Since, $\Q\left(\sqrt{2}+\sqrt{3}+\sqrt{5}\right)$ is the fixed field of some subgroup of the Galois group, it is either $\Q$ or $\Q\left(\sqrt{2}+\sqrt{3}+\sqrt{5}\right)$. Clearly, $\Q\left(\sqrt{2}+\sqrt{3}+\sqrt{5}\right)\not=\Q$, hence \[\Q\left(\sqrt{2},\sqrt{3},\sqrt{5}\right)=\Q\left(\sqrt{2}+\sqrt{3}+\sqrt{5}\right)\qedhere\]
    \end{Solution}
  \end{enumerate}

\end{document}

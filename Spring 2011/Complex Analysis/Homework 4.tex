\documentclass[12pt,leqno]{article}

\usepackage{graphicx,color,amsmath,amsfonts,amssymb,amscd,amsthm,amsbsy,upref}

\title{Complex Analysis\\\large Homework 3}
\date{February 9, 2011}
\author{Khalid Hourani}

\headheight=14.5pt
\textheight=8.5truein
\textwidth=6.0truein
\hoffset=-.5truein
\voffset=-.5truein
\pagestyle{plain}
\footskip=36pt

\theoremstyle{definition}
\newtheorem{thm}{Theorem}
\newtheorem{hthm}[thm]{*Theorem}
\newtheorem{lem}[thm]{Lemma}
\newtheorem{cor}[thm]{Corollary}
\newtheorem{prop}[thm]{Proposition}
\newtheorem{con}[thm]{Conjecture}
\newtheorem{exer}[thm]{Exercise}
\newtheorem{bpe}[thm]{Blank Paper Exercise}
\newtheorem{apex}[thm]{Applications Exercise}
\newtheorem{ques}[thm]{Question}
\newtheorem{scho}[thm]{Scholium}
\newtheorem*{Exthm}{Example Theorem}
\newtheorem*{Thm}{Theorem}
\newtheorem*{Con}{Conjecture}
\newtheorem*{Axiom}{Axiom}

\newtheorem*{Ex}{Example}
\newtheorem*{Def}{Definition}
\newtheorem*{Lem}{Lemma}

\newcommand{\lcm}{\operatorname{lcm}}
\newcommand{\ord}{\operatorname{ord}}
\newcommand{\re}{\operatorname{Re}}
\newcommand{\im}{\operatorname{Im}}
\newcommand{\tr}{\operatorname{tr}}
\newcommand{\Z}{\mathbb{Z}}
\newcommand{\Q}{\mathbb{Q}}
\newcommand{\N}{\mathbb{N}}
\newcommand{\R}{\mathbb{R}}
\newcommand{\C}{\mathbb{C}}
\newcommand{\F}{\mathbb{F}}
\newcommand{\w}{\omega}
\newcommand{\Part}{\center\textbf}
\renewcommand{\labelenumi}{\textbf{\arabic{enumi}.}}
\renewcommand{\labelenumii}{\textbf{(\alph{enumii})}}
\newenvironment{Proof}{\begin{proof}[\textnormal{\textbf{Proof}}]}{\end{proof}}
\newenvironment{Solution}{\begin{proof}[\textnormal{\textbf{Solution}}]}{\end{proof}}
\def\pfrac#1#2{{\left(\frac{#1}{#2}\right)}}
\begin{document}
 \begin{titlepage}
  \maketitle
 \end{titlepage}
\clearpage\mbox{}\clearpage

\setcounter{page}{1}
\begin{enumerate}
 \item Find all circles that are orthogonal to $|z|=1$ and $|z-1|=4$.
  \begin{Solution}
   
  \end{Solution}
 \item All maps considered here are M\"{o}bius transformations. We say that $T$ is \textit{conjugate} to $T_0$ if there exists a transformations $S$ such that $S^{-1}TS=T_0$, that is, $T$ maps $Sz$ to $S\w$ whenever $T_0$ maps $z$ to $\w$. Show that
  \begin{enumerate}
   \item if $T$ has exactly two fixed points, then $T$ is conjugate to a dilation $z\mapsto\lambda z$, $\lambda\in\C$.
   \item if $T$ has exactly one fixed point, then $T$ is conjugate to the translation $z\mapsto z+1$.
  \end{enumerate}
  \begin{Proof}\indent
  \begin{enumerate}
   \item Suppose $z_1,z_2$ are fixed points of the map $T$. Let \[S(z)=\frac{z-z_1}{z-z_2}\] We see that $STS^{-1}$ has fixed points at $0$ and $\infty$, hence is a dilation.
   \item Suppose $z_1$ is the only fixed point of the map $T$. Let \[S(z)=\frac{1}{z-z_1}\] We see that $STS^{-1}$ has no fixed points, hence is a translation. In fact, $STS^{-1}(0)=1$, so the translation is $z\mapsto z+1$.
  \end{enumerate}
  \end{Proof}
   \item 
   \item 
   \item 
   \item Using the curve $\gamma:[-1,1]:\C$, defined by $\gamma(t)=1-t^2+it$, compute \[I_1=\int_{\gamma}\overline{z}dz,\hskip.5in I_2=\int_{\gamma}\frac{dz}{z},\hskip.5in I_3=\int_{\gamma}\frac{dz}{z^3}.\]
    \begin{Solution}\indent
     \begin{itemize}
      \item \begin{align*}I_1=\int_{\gamma}\overline{z}dz=\int_{-1}^1\overline{\gamma(t)}\gamma'(t)dt&=\int_{-1}^1(1-t^2-it)(-2t+i)dt\\&=\int_{-1}^12t^3-t+i\left(t^2+1\right)dt\\&=\frac{t^4}{2}-\frac{t^2}{2}+i\left(\frac{t^3}{3}+t\right)\Big |_{-1}^1\\&=\frac{8}{3}i\end{align*}
      \item \begin{align*}I_2=\int_{\gamma}\frac{1}{z}dz=\int_{-1}^1\frac{\gamma'(t)}{\gamma(t)}dt&=\log(\gamma(t))\Big|_{-1}^1\\&=\log(i)-\log(-i)\\&=i\frac{\pi}{2}-i\frac{-\pi}{2}\\&=\pi i\end{align*}
      \item Let $u=\gamma(t)$. Then $du=\gamma'(t)dt$, hence \begin{align*}I_3=\int_{\gamma}\frac{dz}{z^3}=\int_{-1}^1\frac{\gamma'(t)}{\gamma(t)^3}dt&=\int_{-i}^i\frac{1}{u^3}du\\&=-\frac{1}{2}\frac{1}{u^2}\Big|_{-i}^i\\&=0\end{align*}\qedhere
     \end{itemize}
    \end{Solution}
   \item Compute the same integrals $I_n$ as in problem 6, but for the curve $\gamma(t)=|t|-1+it$, with $t\in[-1,1]$.
    \begin{Solution}\indent
     \begin{itemize}
      \item \begin{align*}I_1=\int_{\gamma}\overline{z}dz=\int_{-1}^1\overline{\gamma(t)}\gamma'(t)dt&=\int_{-1}^0\overline{\gamma(t)}\gamma'(t)dt+\int_0^1\overline{\gamma(t)}\gamma'(t)dt\\&=\int_{-1}^0(-t-1-it)(-1+i)dt\\&+\int_0^1(t-1+it)(1+i)dt\\&=-2i\end{align*}
      \item \begin{align*}I_2=\int_{\gamma}\frac{1}{z}dz=\int_{-1}^1\frac{\gamma'(t)}{\gamma(t)}dt&=\log(\gamma(t))\Big|_{-1}^1\\&=\log(i)-\log(-i)\\&=i\frac{\pi}{2}-i\frac{-\pi}{2}\\&=\pi i\end{align*}
      \item Let $u=\gamma(t)$. Then $du=\gamma'(t)dt$, hence \begin{align*}I_3=\int_{\gamma}\frac{dz}{z^3}=\int_{-1}^1\frac{\gamma'(t)}{\gamma(t)^3}dt&=\int_{-i}^i\frac{1}{u^3}du\\&=-\frac{1}{2}\frac{1}{u^2}\Big|_{-i}^i\\&=0\end{align*}\qedhere
     \end{itemize}
    \end{Solution}
   \item 
    \begin{Proof}
     For any $z_1,z_2\in D$, if the segment $\overline{z_1z_2}$ lies in $D$, take $\gamma:[0,1]\to\C$ with $\gamma=(1-t)z_1+tz_2$, else take $\gamma$ a curve in $D$ connecting $z_1$ and $z_2$ which, in particular, is parallel to $z=it$ for $t\leq1/2$ and parallel to $z=t$ for $t>1/2$, i.e., $\gamma$ is composed of the top and right edges in $D$ of the rectangle whose diagonal is the segment $\overline{z_1z_2}$. In the latter case we have a rectangle whose diagonal length is $|z_1-z_2|$. We shall show that the arclength of $\gamma$, $\mathcal{L}(\gamma)$, is less than $\sqrt{2}|z_1-z_2|$: write $\gamma=\gamma_x+i\gamma_y$, where $\gamma_x=\re(\gamma)$ and $\gamma_y=\im(\gamma)$. Let $X$ denote the arclength of $\gamma_x$, $Y$ the arclength of $\gamma_y$, and $Z$ the distance $|z_1-z_2|$. By definition, $X^2+Y^2=Z^2$, hence \[2XY=XY+XY<X^2+Y^2=Z^2\] Then \[(X+Y)^2=X^2+Y^2+2XY<2Z^2\] hence $X+Y<\sqrt{2}Z$. In either case, we have \begin{align*}\left|f(z_2)-f(z_1)\right|&=\left|\int_{z_1}^{z_2}f'(\gamma(t))\gamma'(t)dt\right|\\&=\left|\int_{\gamma}f'(z)dz\right|\\&\leq\int_{\gamma}|f'(z)|dz\\&\leq\mathcal{L}(\gamma)\leq\sqrt{2}|z_1-z_2|\end{align*}
    \end{Proof}
\end{enumerate}


\end{document}

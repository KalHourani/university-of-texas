\documentclass[12pt,leqno]{book}

\usepackage{fancyhdr,graphicx,color,amsmath,amsfonts,amssymb,amscd,amsthm,amsbsy,upref}

\title{Complex Analysis\\\large Homework 1}
\date{January 26, 2011}
\author{Khalid Hourani}

\headheight=14.5pt
\textheight=8.5truein
\textwidth=6.0truein
\hoffset=-.5truein
\voffset=-.5truein
\pagestyle{plain}
\lhead[ ]{ }\lfoot[\large\textbf{\thepage}]{\footnotesize\rightmark}
\chead[ ]{ }\cfoot[ ]{ }
\rhead[ ]{ }\rfoot[\footnotesize\leftmark]{\large\textbf{\thepage}}
\footskip=36pt

\newcommand{\question}[2] {\vspace{.25in}\noindent\fbox{#1} #2 \vspace{.10in}}
\renewcommand{\part}[1] {\vspace{.10in} {\bf (#1)}}
\renewcommand{\headrulewidth}{0.0pt}
\renewcommand{\footrulewidth}{0.04pt}
\theoremstyle{definition}
\newtheorem{thm}{Theorem}
\newtheorem{hthm}[thm]{*Theorem}
\newtheorem{lem}[thm]{Lemma}
\newtheorem{cor}[thm]{Corollary}
\newtheorem{prop}[thm]{Proposition}
\newtheorem{con}[thm]{Conjecture}
\newtheorem{exer}[thm]{Exercise}
\newtheorem{bpe}[thm]{Blank Paper Exercise}
\newtheorem{apex}[thm]{Applications Exercise}
\newtheorem{ques}[thm]{Question}
\newtheorem{scho}[thm]{Scholium}
\newtheorem*{Exthm}{Example Theorem}
\newtheorem*{Thm}{Theorem}
\newtheorem*{Con}{Conjecture}
\newtheorem*{Axiom}{Axiom}

\newtheorem*{Ex}{Example}
\newtheorem*{Def}{Definition}
\newtheorem*{Lem}{Lemma}

\newcommand{\lcm}{\operatorname{lcm}}
\newcommand{\ord}{\operatorname{ord}}
\newcommand{\re}{\operatorname{Re}}
\newcommand{\im}{\operatorname{Im}}
\newcommand{\Z}{\mathbb{Z}}
\newcommand{\Q}{\mathbb{Q}}
\newcommand{\N}{\mathbb{N}}
\newcommand{\R}{\mathbb{R}}
\newcommand{\C}{\mathbb{C}}
\newcommand{\F}{\mathbb{F}}
\newcommand{\w}{\omega}
\newcommand{\Part}{\center\textbf}
\renewcommand{\labelenumi}{\textbf{\arabic{enumi}.}}
\renewcommand{\labelenumii}{\textbf{(\alph{enumii})}}
\newenvironment{Proof}{\begin{proof}[\textnormal{\textbf{Proof}}]}{\end{proof}}
\newenvironment{Solution}{\begin{proof}[\textnormal{\textbf{Solution}}]}{\end{proof}}
\def\pfrac#1#2{{\left(\frac{#1}{#2}\right)}}
\begin{document}
 \begin{titlepage}
  \maketitle
 \end{titlepage}

\begin{enumerate}
 \item If $S$ is a metric space with distance function $d(x,y)$, show that $S$ with the distance function $\delta(x,y)=d(x,y)/[1+d(x,y)]$ is also a metric space. The latter space is bounded in the sense that all distances lie under a fixed bound.
  \begin{Proof}
   We see that $\delta$ is a function which obviously satisfies the following properties:
  \begin{itemize}
   \item $\delta(x,y)=\delta(y,x)$
   \item $\delta(x,y)=\delta(y,x)$
   \item $\delta(x,y)=0$ if and only if $x=y$
  \end{itemize}
 We must merely show the triangle inequality. Let $a=d(x,z)$, $b=d(y,z)$ and $c=d(x,y)$. Observe that \[\delta(x,z)=a/[1+a],\hskip.15in \delta(y,z)=b/[1+b],\hskip.15in \text{and }\delta(x,y)=c/[a+c]\] By the triangle inequality, we have $c\leq a+b$, hence \[c\leq a+b+2ab+abc\] Adding $ac+bc+abc$ to both sides, we have \[c+ac+bc+abc\leq a+b+2ab+2abc+ac+bc+2abc\] Equivalently, we have \[(a(1+b)(1+c)+b(1+a)(1+c)\leq c(1+a)(1+b)\] Dividing through by $1+a$, $a+b$ and $1+c$ we have \[\frac{1}{1+a}+\frac{b}{1+b}\leq\frac{c}{1+c}\] hence \[\delta(x,y)\leq\delta(x,z)+\delta(z,y)\qedhere\] 
  \end{Proof}
 \item Classify each of the following sets as open, closed, or neither open nor closed.
  \begin{enumerate}
   \item $A=\{z\in\C:-\pi<\im z\leq\pi\}$
   \item $B=\{z\in\C:1<|z|<2\}$
   \item $C=\{z\in\C:|\re z|+|\im z|\leq1\}$
   \item $D=\{z\in\C:0<\max\{x,y\}\leq1\}$
   \item $E=\{z\in\C:y>x^2\}$
   \item $F=\{z\in\C:\re z\text{ and }\im z\text{ are rational}\}$
  \end{enumerate}
\begin{Proof}\indent
 \begin{enumerate}
  \item This set is neither open nor closed. To see that it is non-open, observe that any open ball centered at $i\pi$ is not strictly contained in $A$. To see that it is not closed, we apply the same argument to show that its closure is not open: any open ball centered at $-i\pi$ is not strictly contained in $A^c$. 
  \item This set is clearly open: for any point $z\in B$, take $\epsilon=\min(\frac{2-|z|}{2},\frac{|z|-1}{2})$. Then the ball of radius epsilon centered at $z$ is contained in $B$.
  \item This is the same set as $\{(x,y)|x+y\leq1\}\subseteq\R^2$ which is clearly closed.
  \item This set is neither open nor closed, as it is merely the union of unit squares with vertex at the origin in regions I, II and IV, with the origin removed.
  \item This set is clearly open.
  \item This set is neither open nor closed, for its closure is in fact all of $\C$ and its interior is empty. 
 \end{enumerate}
\end{Proof}
 \item Determine the boundary, closure, and interior of each of the sets listed in the previous exercise.
  \begin{enumerate}
   \item The closure of this set is $\{z\in\C:-\pi\leq\im z\leq\pi\}$ and its interior is $\{z\in\C:-\pi<\im z<\pi\}$ hence its boundary is the pair of lines $\im z=\pm\pi$. 
   \item This set is open, hence equal to its interior. Its closure is $\{z\in\C:1\leq|z|\leq2\}$ hence its boundary is simply the annulus $\{z:|z|=1\text{ or }|z|=2\}$.
   \item Since this set is closed, it is equal to its closure. Its interior is $\{z\in\C:|\re z|+|\im z|<1\}$ hence its boundary is the diamond formed by the four lines \begin{align*}\{\re(z)+\im(z)=1\}\\\{\re(z)-\im(z)=1\}\\\{\re(z)+\im(z)=-1\}\\\{\re(z)-\im(z)=-1\}\end{align*}
   \item This set is closed, hence equal to its closure. Its interior is merely $\{z\in\C:0<\max\{x,y\}<1\}$.
   \item This set is open, hence equal to its interior. The closure of this set is $\{z\in\C:y\geq x^2\}$. Thus, the boundary is the parabola $y=x^2$. 
   \item This sets closure is $\C$ and its interior is empty, hence its boundary is $\C$. 
  \end{enumerate}
 \item Prove that the closure of a connected set is connected.
  \begin{Proof}
   We prove the contrapositive: suppose $A$ is a set whose closure is disconnected, say $\overline{A}=U\cup V$ with $U$, $V$ open.  However, $\overline{\overline{A}}=\overline{A}=\overline{U}\cup\overline{V}$, hence $U=\overline{U}$ and $V=\overline{V}$. In particular, $\overline{A}$ is the union of disjoint closed sets $U$ and $V$. We see that $A=(U\cap A)\cup(V\cap A)$ and that $U\cap A$ and $V\cap A$ are both nonempty and open. Thus, $A$ is disconnected.
  \end{Proof}
 \item Show that the union of two regions is a region if and only if they share a common point.
  \begin{Proof}
   In fact, if the union of two regions is a region, they must share a point, for otherwise their union could be written as the union of disjoint open sets, and would be disconnected. For the converse, suppose regions $U$ and $V$ share a common point. In order to reach a contradiction, suppose $U\cup V$ is disconnected, i.e., that $U\cup V=(A\cup B)$ with $A$, $B$ open and $A\cap B=\emptyset$. Since $A\cap B\not=\emptyset$, $U$ and $V$ cannot have elements in both $A$ and $B$. In particular, we can write $U\subseteq A$ and $V\subseteq B$. However, this contradicts $A\cap B=\emptyset$, hence $U\cup V$ must be connected. Since both $U$ and $V$ are regions, their union is both nonempty and open, hence $U\cup V$ is a region.
  \end{Proof}
 \item If $z=x+iy$ ($x$ and $y$) real, find the real and imaginary parts of \[z^4, \frac{1}{z}, \frac{z-1}{z+1}, \frac{1}{z^2}\]
  \begin{Solution}\indent
   \begin{itemize}
    \item The binomial theorem, while noting $i^0=1$, $i^1=i$, $i^2=-1$, $i^3=-i$, and that $\{i^n:n\in\Z\}$ forms a cyclic group of order 4, yields: \[z^4=(x+iy)^4=x^4+y^4-6x^2y^2+i(4x^3y-4xy^3)\] hence the real part and imaginary part of $z^4$ are $x^4+y^4-6x^2y$ and $4x^3-3xy^3$, respectively.
    \item We rewrite $\frac{1}{x+iy}$ as follows: \[\frac{1}{x+iy}=\frac{x-iy}{(x+iy)(x-iy)}=\frac{x-iy}{x^2-y^2}\] hence the real part and imaginary part of $\frac{1}{z}$ are \[\frac{x}{x^2-y^2}\text{ and }\frac{-y}{x^2-y^2}\] respectively.
    \item Similarly, we have \begin{align*}\frac{z-1}{z+1}&=\frac{(z-1)(\overline{z}+1)}{(z+1)(\overline{z}+1)}\\&=\frac{z\overline{z}+(z-\overline{z})-1}{z\overline{z}+(z+\overline{z})+1}\\&=\frac{|z|+2i\im z-1}{|z|+2\re{z}+1}\end{align*} Hence the real part and imaginary part of $\frac{z-1}{z+1}$ are \[\frac{|z|-1}{|z|+2x+1}\text{ and }\frac{2y}{|z|+2x+1}\] respectively.
    \item We begin by writing $z^2=(x+iy)^2=x^2-y^2+2ixy$. Then \[\frac{1}{z^2}=\frac{\overline{z}^2}{z^2\overline{z}^2}=\frac{\overline{z}^2}{|z|^2}\] Hence the real part and imaginary part of $\frac{1}{z^2}$ are \[\frac{x^2-y^2}{|z|^2}\text{ and }\frac{-2xy}{|z|^2}\] respectively.
   \end{itemize}
  \end{Solution}
 \item Prove that \[\left|\frac{z-\w}{1-\overline{\w}z}\right|<1\tag{*}\] if $|\w|<1$ and $|z|<1$. For what $\w$ and $z$ is the left hand side of $(*)$ equal to 1?
\end{enumerate}

\end{document}

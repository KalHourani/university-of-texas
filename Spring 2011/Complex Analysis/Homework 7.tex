\documentclass[12pt,leqno]{article}

\usepackage{graphicx,color,amsmath,amsfonts,amssymb,amscd,amsthm,amsbsy,upref}

\title{Complex Analysis\\\large Homework 7}
\date{March 23, 2011}
\author{Khalid Hourani}

\headheight=14.5pt
\textheight=8.5truein
\textwidth=6.0truein
\hoffset=-.5truein
\voffset=-.5truein
\pagestyle{plain}
\footskip=36pt

\theoremstyle{definition}
\newtheorem{thm}{Theorem}
\newtheorem{hthm}[thm]{*Theorem}
\newtheorem{lem}[thm]{Lemma}
\newtheorem{cor}[thm]{Corollary}
\newtheorem{prop}[thm]{Proposition}
\newtheorem{con}[thm]{Conjecture}
\newtheorem{exer}[thm]{Exercise}
\newtheorem{bpe}[thm]{Blank Paper Exercise}
\newtheorem{apex}[thm]{Applications Exercise}
\newtheorem{ques}[thm]{Question}
\newtheorem{scho}[thm]{Scholium}
\newtheorem*{Exthm}{Example Theorem}
\newtheorem*{Thm}{Theorem}
\newtheorem*{Con}{Conjecture}
\newtheorem*{Axiom}{Axiom}

\newtheorem*{Ex}{Example}
\newtheorem*{Def}{Definition}
\newtheorem*{Lem}{Lemma}

\newcommand{\lcm}{\operatorname{lcm}}
\newcommand{\ord}{\operatorname{ord}}
\newcommand{\re}{\operatorname{Re}}
\newcommand{\im}{\operatorname{Im}}
\newcommand{\tr}{\operatorname{tr}}
\newcommand{\Z}{\mathbb{Z}}
\newcommand{\Q}{\mathbb{Q}}
\newcommand{\N}{\mathbb{N}}
\newcommand{\R}{\mathbb{R}}
\newcommand{\C}{\mathbb{C}}
\newcommand{\F}{\mathbb{F}}
\newcommand{\w}{\omega}
\newcommand{\Part}{\center\textbf}
\renewcommand{\labelenumi}{\textbf{\arabic{enumi}.}}
\renewcommand{\labelenumii}{\textbf{(\alph{enumii})}}
\newenvironment{Proof}{\begin{proof}[\textnormal{\textbf{Proof}}]}{\end{proof}}
\newenvironment{Solution}{\begin{proof}[\textnormal{\textbf{Solution}}]}{\end{proof}}
\def\pfrac#1#2{{\left(\frac{#1}{#2}\right)}}
\begin{document}
 \begin{titlepage}
  \maketitle
 \end{titlepage}
\clearpage\mbox{}\clearpage

\setcounter{page}{1}
\begin{enumerate}
 \item Let $f$ be analytic on $\C\setminus\{0\}$ and satisfy $f(z)=f(1/z)$ for all nonzero $z\in\C$. Show that there exists an entire function $g$ such that $f(z)=g(z+1/z)$ for all nonzero $z\in\C$. 
  \begin{Proof}
   Let $g(z)=f\left(\frac{z+\sqrt{z^2-4}}{2}\right)$. Clearly, $g$ is entire. Set $\w=z+1/z$. We see that \[z^2-\w z+1=0\] hence \[z=\frac{\w\pm\sqrt{\w^2-4}}{2}.\] Further, \begin{align*}\left(\frac{\w+\sqrt{\w^2-4}}{2}\right)\left(\frac{\w-\sqrt{\w^2-4}}{2}\right)&=\frac{\w^2-\w^2+4}{4}\\&=1\end{align*} Thus, if $z+1/z=\w$, $\frac{\w+\sqrt{\w^2-4}}{2}=z$ or $\frac{\w+\sqrt{\w^2-4}}{2}=\frac{1}{z}$. Since $f(z)=f(1/z)$, we have \begin{align*}g\left(z+1/z\right)=g(\w)&=f\left(\frac{\w+\sqrt{\w^2-4}}{2}\right)\\&=f(z)\text{ or }f(1/z)\\&=f(z)\qedhere\end{align*}
  \end{Proof}
 \item Show that $f:\C_{\infty}\to\C_{\infty}$ is analytic if and only if $f$ is a rational function.
\end{enumerate}


\end{document}

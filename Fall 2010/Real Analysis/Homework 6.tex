\documentclass[12pt,leqno]{book}

\usepackage{fancyhdr,graphicx,color,amsmath,amsfonts,amssymb,amscd,amsthm,amsbsy,upref}

\title{Real Analysis\\\large Homework 6}
\date{October 15, 2010}
\author{Khalid Hourani}
\headheight=14.5pt
\textheight=8.5truein
\textwidth=6.0truein
\hoffset=-.5truein
\voffset=-.5truein
\pagestyle{fancy}
\lhead[ ]{ }\lfoot[\large\textbf{\thepage}]{\footnotesize\rightmark}
\chead[ ]{ }\cfoot[ ]{ }
\rhead[ ]{ }\rfoot[\footnotesize\leftmark]{\large\textbf{\thepage}}
\footskip=36pt

\newcommand{\question}[2] {\vspace{.25in}\noindent\fbox{#1} #2 \vspace{.10in}}
\renewcommand{\part}[1] {\vspace{.10in} {\bf (#1)}}
\renewcommand{\headrulewidth}{0.0pt}
\renewcommand{\footrulewidth}{0.04pt}
\theoremstyle{definition}
\newtheorem{thm}{Theorem}
\newtheorem{hthm}[thm]{*Theorem}
\newtheorem{lem}[thm]{Lemma}
\newtheorem{cor}[thm]{Corollary}
\newtheorem{prop}[thm]{Proposition}
\newtheorem{con}[thm]{Conjecture}
\newtheorem{exer}[thm]{Exercise}
\newtheorem{bpe}[thm]{Blank Paper Exercise}
\newtheorem{apex}[thm]{Applications Exercise}
\newtheorem{ques}[thm]{Question}
\newtheorem{scho}[thm]{Scholium}
\newtheorem*{Exthm}{Example Theorem}
\newtheorem*{Thm}{Theorem}
\newtheorem*{Con}{Conjecture}
\newtheorem*{Axiom}{Axiom}

\newtheorem*{Ex}{Example}
\newtheorem*{Def}{Definition}
\newtheorem*{Lem}{Lemma}

\newcommand{\lcm}{\operatorname{lcm}}
\newcommand{\ord}{\operatorname{ord}}
\newcommand{\Z}{\mathbb{Z}}
\newcommand{\Q}{\mathbb{Q}}
\newcommand{\N}{\mathbb{N}}
\newcommand{\R}{\mathbb{R}}
\newcommand{\C}{\mathbb{C}}
\newcommand{\Part}{\center\textbf}
\renewcommand{\labelenumi}{\textbf{(\arabic{enumi})}}
\renewcommand{\labelenumii}{\textbf{\alph{enumii})}}
\newenvironment{Proof}{\begin{proof}[\textnormal{\textbf{Proof}}]}{\end{proof}}
\newenvironment{Solution}{\begin{proof}[\textnormal{\textbf{Solution}}]}{\end{proof}}
\def\pfrac#1#2{{\left(\frac{#1}{#2}\right)}}
\begin{document}
\begin{titlepage}
 \maketitle 
\end{titlepage}

\begin{enumerate}
 \item Let $1<p<\infty$ and $f \in L^p(0,+\infty)$.  Define \[F(x)=\frac{1}{x}\int_0^xf(t)dt,\;x>0\]
  \begin{enumerate} 
   \item Prove Hardy's Inequality $$||F||_p \leq \frac p{p-1}||f||_p$$ 
   \item Prove that equality holds above if and only if $f = 0$ almost everywhere.
   \item Prove that the constant $\frac p{p-1}$ cannot be replaced with a smaller one
   \item If $f>0$ and $f \in L^1$, prove that $F \not\in L^1$ (Observe that when $f$ is continuous, we have $xF' = f-F$).
  \end{enumerate}

\begin{Proof}
 \begin{enumerate}
  \item By H\"{o}lder's Inequality, we have \[\left|\int_0^af(t)dt\right|\leq\int_0^a|f(t)|dt\leq\left(\int_0^af^p(t)dt\right)^{1/p}\left(\int_0^adt\right)^{1/q}=\|f\|_{L^p}a^{1/q}\] hence \begin{align*}\left|\int_0^af(t)dt\right|^p&\leq\|f\|^p_{L^p}a^{p-1}\\a^{1-p}\left|\int_0^af(t)dt\right|^p&\leq\|f\|_{L^p}\end{align*} Thus, \[\lim_{a\to0}a^{1-p}\left|\int_0^af(t)dt\right|^p=0\] 

Let $\overline{F}(x)=xF(x)$ and note that $\overline{F}$ is continuous and has derivative equal to $f$ almost everywhere. Integrating by parts, we see that \begin{align*}\int_a^bF^p(x)dx&=\int_a^b\frac{\overline{F}^p(x)}{x^p}dx\\&=\frac{a^{1-p}\overline{F}^p(x)}{p-1}-\frac{b^{1-p}\overline{F}^p(b)}{p-1}+\frac{p}{1-p}\int_a^b\overline{F}^{p-1}(x)f(x)x^{1-p}dx\end{align*} As $a\to0$, this becomes \begin{align*}\int_0^bF^p(x)dx&=\frac{-b^{1-p}\overline{F}^p(b)}{p-1}+\frac{p}{p-1}\int_0^b\overline{F}^{p-1}(x)f(x)x^{1-p}dx\\&\leq\frac{p}{p-1}\int_0^b\overline{F}^{p-1}(x)f(x)x^{1-p}dx\\&\leq\frac{p}{p-1}\left(\int_0^bF^p(x)dx\right)^{1/q}\left(\int_0^bf^p(x)dx\right)^{1/p}\end{align*} Thus, we have \[\left(\int_0^bF^p(x)dx\right)^{1/p}\leq\frac{p}{p-1}\left(\int_0^bf^p(x)dx\right)^{1/p}\leq\frac{p}{p-1}\left(\int_0^{\infty}f^p(x)dx\right)^{1/p}\] However, $b$ was chosen to be arbitrary, so \[\|F\|_{L^p}\leq\frac{p}{p-1}\|f\|_{L^p}\]
 \end{enumerate}

\end{Proof}

 \item For $1<p<\infty$ and $f \in L^p(\R^n)$, show that \[\int_{\R^n}|f|^pdx=p\int_0^{\infty} \alpha^{p-1}\lambda(\alpha)d\alpha\] where $\lambda(\alpha)=|\{x:|f(x)|\geq\alpha\}|$. 

\begin{Proof}
 We begin by showing the result in the case where $f$ is a characterstic function. Write $f(x)=\chi_E(x)$. Clearly, $\lambda(\alpha)$ is just 0 whenever $\alpha>1$ and $|E|$ elsewhere. We see that \begin{align*}p\int_0^{\infty}\alpha^{p-1}\lambda(\alpha)d\alpha&=p\int_0^1\alpha^{p-1}|E|d\alpha\\&=|E|\\&=\int_{\R^n}|f|^pdx\end{align*} The result follows in the case where $f$ is simple by noting that if $E_1,E_2,\hdots E_k$ are disjoint sets, then \[\left(\sum_{i=1}^ka_i\chi_{E_i}(x)\right)^p=\sum_{i=1}^ka_i^p\chi_{E_i}(x)\] 

Now, take $f\in L^p(\R^n)$ and let $f_k$ be a sequence of simple functions in $L^p(\R)$ with $f_k\to f$. Observe that \begin{align*}\int_{\R^n}|f|^pdx=\lim_{k\to\infty}\int_{\R^n}|f_k|^pdx=\lim_{k\to\infty}p\int_0^{\infty}\alpha^{p-1} \lambda_k(\alpha)d\alpha\end{align*} where $\lambda_k(\alpha)=|\{x:f_k(x)|\geq\alpha\}|$. Clearly, $\lambda_k\to\lambda$, where $\lambda(\alpha)=|\{x:f_k(x)|\geq\alpha\}|$, hence \[\int_{\R^n}|f|^pdx=p\int_0^{\infty} \alpha^{p-1}\lambda(\alpha)d\alpha\qedhere\]
\end{Proof}

 \item Let $f \in L^p$, $g \in L^q$ where $\frac1p + \frac1q = 1$.  Prove that their convolution is a function in $C_0(\R^n)$ (a continuous function vanishing at infinity). 
 \item (\textit{variant of Vitali's covering lemma})  Let $E$ (not necessarily measurable) be covered in the Vitali sense by a family of balls $\mathcal B$, and suppose we have $0< |E|_e< \infty$.  Show that for any $\eta >0$, there exists a disjoint, countable subcollection $\{B_k\}_{k=1}^\infty \subset \mathcal B$ Such that $$|E \backslash \cup_{k=1}^\infty B_k|_e = 0\text{  and  }\Sigma_{k=1}^\infty | B_k| \leq (1+\eta)|E|_e$$
\end{enumerate}
\end{document}
\documentclass[12pt,leqno]{book}

\usepackage{fancyhdr,graphicx,color,amsmath,amsfonts,amssymb,amscd,amsthm,amsbsy,upref}

\title{Real Analysis\\\large Homework 5}
\date{September 30, 2010}
\author{Khalid Hourani}
\headheight=14.5pt
\textheight=8.5truein
\textwidth=6.0truein
\hoffset=-.5truein
\voffset=-.5truein
\pagestyle{fancy}
\lhead[ ]{ }\lfoot[\large\textbf{\thepage}]{\footnotesize\rightmark}
\chead[ ]{ }\cfoot[ ]{ }
\rhead[ ]{ }\rfoot[\footnotesize\leftmark]{\large\textbf{\thepage}}
\footskip=36pt

\newcommand{\question}[2] {\vspace{.25in}\noindent\fbox{#1} #2 \vspace{.10in}}
\renewcommand{\part}[1] {\vspace{.10in} {\bf (#1)}}
\renewcommand{\headrulewidth}{0.0pt}
\renewcommand{\footrulewidth}{0.04pt}
\theoremstyle{definition}
\newtheorem{thm}{Theorem}
\newtheorem{hthm}[thm]{*Theorem}
\newtheorem{lem}[thm]{Lemma}
\newtheorem{cor}[thm]{Corollary}
\newtheorem{prop}[thm]{Proposition}
\newtheorem{con}[thm]{Conjecture}
\newtheorem{exer}[thm]{Exercise}
\newtheorem{bpe}[thm]{Blank Paper Exercise}
\newtheorem{apex}[thm]{Applications Exercise}
\newtheorem{ques}[thm]{Question}
\newtheorem{scho}[thm]{Scholium}
\newtheorem*{Exthm}{Example Theorem}
\newtheorem*{Thm}{Theorem}
\newtheorem*{Con}{Conjecture}
\newtheorem*{Axiom}{Axiom}

\newtheorem*{Ex}{Example}
\newtheorem*{Def}{Definition}
\newtheorem*{Lem}{Lemma}

\newcommand{\lcm}{\operatorname{lcm}}
\newcommand{\ord}{\operatorname{ord}}
\newcommand{\Z}{\mathbb{Z}}
\newcommand{\Q}{\mathbb{Q}}
\newcommand{\N}{\mathbb{N}}
\newcommand{\R}{\mathbb{R}}
\newcommand{\C}{\mathbb{C}}
\newcommand{\Part}{\center\textbf}
\renewcommand{\labelenumi}{\textbf{(\arabic{enumi})}}
\renewcommand{\labelenumii}{\textbf{\alph{enumii})}}
\newenvironment{Proof}{\begin{proof}[\textnormal{\textbf{Proof}}]}{\end{proof}}
\newenvironment{Solution}{\begin{proof}[\textnormal{\textbf{Solution}}]}{\end{proof}}
\def\pfrac#1#2{{\left(\frac{#1}{#2}\right)}}
\begin{document}
\begin{titlepage}
 \maketitle 
\end{titlepage}

\begin{enumerate}
 \item Construct a non-negative funciton $f\in L^1{\R}$ which is essentially unbounded in every interval (i.e. for any interval $I$ we have $\text{essup}f=+\infty$).

\begin{Proof}
 Let $g(x)=x^{-1/2}\chi_{[0,1]}$. Let $f:\N\to\Q$ be a bijection, and define $a_n=f(n)$. Now, let \[f(x)=\sum_{n=1}^{\infty}2^{-n}g(x-a_n)\] Then \begin{align*}\int|f(x)|dx&=\sum_{n=1}^{\infty}2^{-n}\int|g(x-a_n)|dx\\&=\sum_{n=1}^{\infty}2^{-n}\int|g(x)|dx\\&=\int|g(x)|dx\end{align*} Then, for any interval $I$, there is an interval $J$ about the origin such that \[\underset{I}{\text{essup}}f=\underset{J}{\text{essup}}g\] Hence \[\underset{I}{\text{essup}}f=+\infty\qedhere\]
\end{Proof}

 \item A few limits with integrals:
  \begin{enumerate}
   \item Find whether $\lim_{n\to\infty}\int_0^1(1-\frac{x}{n})^ne^{\frac{x}{2}}dx$ exists and if it does find its value.
   \item (Riemann-Lebesgue Lemma) Let $f\in L^1(\R)$. Compute \[\lim_{n\to\infty}\int_{\R}f(x)\cos(nx)dx\]
   \item Let $E\subseteq(0,2\pi)$ and $u_n$ some arbitrary sequence of real numbers. Compute \[\lim_{n\to\infty}\int_E(\cos(nx+u_n))^2dx\]
  \end{enumerate}

\begin{Proof}\indent
 \begin{enumerate}
  \item Write $f_n(x)=(1-\frac{x}{n})^ne^{\frac{x}{2}}$ and observe that $f_n(x)$ is dominated by (and further has a limit of) $e^{\frac{3x}{2}}$. By dominated convergence, \begin{align*}\lim_{n\to\infty}\int_0^1\left(1-\frac{x}{n}\right)^n\cos(nx)dx&=\int_0^1e^{\frac{3x}{2}}\\&=\frac{2}{3}\left(e^{\frac{3}{2}}-1\right)\end{align*}

  \item We first observe that \[\lim_{n\to\infty}\int_I\cos(nx)dx=0\] for any compact interval $I$. Further, for any step function $f(x)=\lambda\chi_I$, \[\lim_{n\to\infty}\int_{\R}f(x)\cos(nx)dx=\lambda\lim_{n\to\infty}\int_I\cos(nx)dx=0\] For any positive function $f$, we have $f_k\to f$ with $f_k$ a sequence of step functions and, by dominated convergence, \begin{align*}\lim_{n\to\infty}\int_{\R}f(x)\cos(nx)dx&=\lim_{n\to\infty}\lim_{k\to\infty}\int_{\R}f_k(x)\cos(nx)dx\\&=\lim_{k\to\infty}\lim_{n\to\infty}\int_{\R}f_k(x)\cos(nx)dx\\&=0\end{align*} Then, for any measurable function $f$, write $f=g-h$ with $g,h$ positive $L_1$ functions. Then \begin{align*}\lim_{n\to\infty}\int_{\R}f(x)\cos(nx)dx&=\lim_{n\to\infty}\int_{\R}g(x)\cos(nx)dx-\lim_{n\to\infty}\int_{\R}h(x)\cos(nx)dx\\&=0\end{align*} which completes the proof.

  \item Recall that \[\cos^2(\theta)=\frac{1+\cos2\theta}{2}\] and so \begin{align*}\int_E\cos^2(nx+u_n)dx&=\frac{1}{2}\int_E1dx+\frac{1}{2}\int_E\cos2\left(nx+u_n\right)dx\\&=\frac{1}{2}|E|+\frac{1}{2}\cos2u_n\int_E\cos2nxdx-\frac{1}{2}\sin2u_n\int_E\sin2nxdx\end{align*} By the Riemann-Lebesgue Lemma, \begin{align*}\int_E\cos2nxdx&=\int_{-\infty}^{\infty}\chi_E(x)\cos2nxdx\to0\text{ as }n\to\infty\\\int_E\sin2nxdx&=\int_{-\infty}^{\infty}\chi_E(x)\sin2nxdx\to0\text{ as }n\to\infty\end{align*} hence \[\int_E\cos^2(nx+u_n)dx\to\frac{1}{2}|E|\text{ as }n\to\infty\qedhere\]

\end{enumerate}
\end{Proof}

 \item Given $f$ measurable in $\R^n$, we define its distribution funciton $\lambda:\R_+\to\R_+$ by \[\lambda(t)=|\{x\in\R:|f|>t\}|\] Prove $f\in L^p(\R)$ if and only if $\displaystyle\sum_{-\infty}^{\infty}2^{kp}\lambda(2^k)<+\infty$.

\begin{Proof}
 On the one hand, if $f\in L^p(\R)$, then $f^p\in L^1(\R)$. Then we can approximate $f^p$ by a Riemann sum which is greater than \[\sum_{-\infty}^{\infty}2^{kp}\lambda(2^k)\] hence this value is finite.

On the other hand, if this value is finite, then, for every $\epsilon>0$, there is a Riemann sum of the function $f^p$ which is less than $\epsilon$, hence $f^p\in L^1(\R)$. Thus, $f\in L^p(\R)$. 
\end{Proof}


 \item First some definitions: A complex-valued function $f:\R^n\to\C$ is \textit{measurable} if its real and imaginary parts are measurable. It is also defined to be in $L^p$ if $|f(x)|$ (a real valued, measurable function) is in $L^p$. Given $f\in L^1(\R)$, with calues in $\C$, the \textit{Fourier transform} of $f$ (traditionally called ``f hat'') is function \[\hat{f}:\R^n\to\R,\hskip.05in \hat{f}(y):=\int_{\R}f(x)e^{-2\pi ix\cdot y}dx\hskip.2in y\in\R^n\] Note: for complex-valued functions we define \[\int_f(x)dx=\int\text{Re}(f(x))dx+i\int\text{Im}(f(x))dx\] With these definitions, prove (assuming $f\in L^1$)
  \begin{enumerate}
   \item $\hat{f}(y)$ is a continuous function in $\R^n$
   \item (Riemann-Lebesgue revisited) As $|y|\to\infty$ we have $\hat{f}(y)\to0$. 
  \end{enumerate}

\begin{Proof}\indent
 \begin{enumerate}
  \item Observe that \begin{align*}\left|\hat{f}(w)-\hat{f}(z)\right|&=\left|\int_{\R}\left(e^{-2\pi ix\cdot w}-e^{-2\pi ix\cdot z}\right)f(x)dx\right|\\&=\left|\int_{\R}e^{-2\pi ix\cdot w}\left(e^{-2\pi ix\cdot(z-w)}-1\right)f(x)d(x)\right|\\&\leq\int_{\R}\left|e^{-2\pi i(z-w)x}-1\right||f(x)|d(x)\\&\leq2\int_{\R}|f(x)|dx\end{align*} So $\hat{f}$ is continuous (in fact uniformly continuous).
  \item This proof follows the same format as \textbf{(2b)}. Simply write \[\hat{f}(y)=\int_{\R}f(x)\cos(2\pi yx)dx+i\int_{\R}f(x)\sin(2\pi yx)dx\] Then as $|y|\to\infty$, $\hat{f}(y)\to0$.
 \end{enumerate}

\end{Proof}

\end{enumerate}
\end{document}
\documentclass[12pt,leqno]{book}

\usepackage{fancyhdr,graphicx,color,amsmath,amsfonts,amssymb,amscd,amsthm,amsbsy,upref}

\title{Real Analysis\\\large Homework 8}
\date{October 28, 2010}
\author{Khalid Hourani}
\headheight=14.5pt
\textheight=8.5truein
\textwidth=6.0truein
\hoffset=-.5truein
\voffset=-.5truein
\pagestyle{fancy}
\lhead[ ]{ }\lfoot[\large\textbf{\thepage}]{\footnotesize\rightmark}
\chead[ ]{ }\cfoot[ ]{ }
\rhead[ ]{ }\rfoot[\footnotesize\leftmark]{\large\textbf{\thepage}}
\footskip=36pt

\newcommand{\question}[2] {\vspace{.25in}\noindent\fbox{#1} #2 \vspace{.10in}}
\renewcommand{\part}[1] {\vspace{.10in} {\bf (#1)}}
\renewcommand{\headrulewidth}{0.0pt}
\renewcommand{\footrulewidth}{0.04pt}
\theoremstyle{definition}
\newtheorem{thm}{Theorem}
\newtheorem{hthm}[thm]{*Theorem}
\newtheorem{lem}[thm]{Lemma}
\newtheorem{cor}[thm]{Corollary}
\newtheorem{prop}[thm]{Proposition}
\newtheorem{con}[thm]{Conjecture}
\newtheorem{exer}[thm]{Exercise}
\newtheorem{bpe}[thm]{Blank Paper Exercise}
\newtheorem{apex}[thm]{Applications Exercise}
\newtheorem{ques}[thm]{Question}
\newtheorem{scho}[thm]{Scholium}
\newtheorem*{Exthm}{Example Theorem}
\newtheorem*{Thm}{Theorem}
\newtheorem*{Con}{Conjecture}
\newtheorem*{Axiom}{Axiom}

\newtheorem*{Ex}{Example}
\newtheorem*{Def}{Definition}
\newtheorem*{Lem}{Lemma}

\newcommand{\lcm}{\operatorname{lcm}}
\newcommand{\ord}{\operatorname{ord}}
\newcommand{\Z}{\mathbb{Z}}
\newcommand{\Q}{\mathbb{Q}}
\newcommand{\N}{\mathbb{N}}
\newcommand{\R}{\mathbb{R}}
\newcommand{\C}{\mathbb{C}}
\newcommand{\Part}{\center\textbf}
\renewcommand{\labelenumi}{\textbf{(\arabic{enumi})}}
\renewcommand{\labelenumii}{\textbf{\alph{enumii})}}
\newenvironment{Proof}{\begin{proof}[\textnormal{\textbf{Proof}}]}{\end{proof}}
\newenvironment{Solution}{\begin{proof}[\textnormal{\textbf{Solution}}]}{\end{proof}}
\def\pfrac#1#2{{\left(\frac{#1}{#2}\right)}}
\begin{document}
\begin{titlepage}
 \maketitle 
\end{titlepage}

\begin{enumerate}
\item Fix $f\in L^1(\R^n)$, show the following:
  \begin{enumerate}
   \item Let first $n=1$, and take $g(x)=\int_0^xf(t)dt$ (recalling the convention that if $x<0$ then $\int_0^x=-\int_x^0$).
    
    Prove that for each $\epsilon>0$ one can find $\delta>0$ such that for \textit{every} sequence $x_1<\hdots<x_N$ such that $\displaystyle\sum_i|x_{i+1}-x_i|<\delta$ we have \[\sum_i|g(x_{i+1})-g(x_i)|<\epsilon\]
    Note: this property is called ``absolute continuity''
   \item (for any $n$) Show that given $\epsilon>0$ one can find $\delta>0$ such that $\forall E$ with $|E|<\delta$ one has $\int_E|f|dx<\epsilon$.
   \item (also for any $n$) Given $\epsilon>0$ one can find $R>0$ such that $\forall E$ for which $|E\cap B_R(0)|=0$ we have $\int_E|f|dx<\epsilon$.
  \end{enumerate}

\begin{Proof}\indent
 \begin{enumerate}
  \item [\textbf{b)}] We show this result for bounded functions: for every $\epsilon>0$ take $\delta=\frac{\epsilon}{\|f\|_{L^{\infty}}}$. If $|E|<\delta$, then \[\int_{E}f(t)dt\leq|E|\|f\|_{L^{\infty}}=\epsilon\] For any $f$ and any $m\in\R$ we can define functions $G_m=f\chi_{\{x|f(x)\leq m\}}$ and $B_m=f\chi_{\{x|f(x)>m\}}$ and write $f=G_m+B_m$. In particular, we shall show that, for $f\in L^1(\R^n)$, for any $\epsilon>0$, there is an $m>0$ such that $\|B_m\|_{L^1(\R^n)}<\epsilon$, from which the result follows: \[\int_Ef=\int_EG_m+\int_EB_m<\frac{\epsilon}{2}+\frac{\epsilon}{2}=\epsilon\] However, the existence of such an $m$ follows from the density of bounded functions in $L^1(\R^n)$.
  \item [\textbf{a)}] From \textbf{b)}, if $|x_N-x_1|=\displaystyle\sum_i|x_{i+1}-x_i|<\delta$, then \begin{align*}\sum_i|g(x_{i+1})-g(x_i)|&=\sum_i\left|\int_{x_i}^{x_{i+1}}f(t)dt\right|\\&\leq\sum_i\int_{x_i}^{x_{i+1}}|f(t)|dt\\&=\int_{x_1}^{x_N}|f(t)|dt\\&<\epsilon\end{align*}
  \item Take $R$ such that $\int_{B_R(0)}|f(x)|dx>\int_{\R^n}-\epsilon$. Then \begin{align*}\int_{B_R(0)}f-\int_{\R^n}f&<\epsilon\\\int_{(B_R(0))^c}f&<\epsilon\end{align*} However $\int_Ef$ is clearly less than $\int_{(B_R(0))^c}f$, hence \[\int_Ef<\epsilon\qedhere\]
 \end{enumerate}

\end{Proof}


\item Take $f\in L^p(\R^n)$, $1<p\leq+\infty$ consider again $g(x)=\int_0^xf(t)dt$.
  \begin{enumerate}
   \item Prove that $g$ is ``H\"{o}lder continuous with exponent $\alpha=1-\frac{1}{p}$,'' that is, there is a constant $C>0$ such that \[|g(x)-g(y)|\leq C|x-y|^{\alpha},\:\forall x,y\in\R\] In fact, show it is enough to take $C=\|f\|_{L^p(\R^1)}$.
   \item Prove for almost every $x\in\R$ that $g'(x)$ exists and that $g'(x)=f(x)$.
  \end{enumerate}

\begin{Proof}
  \begin{enumerate}
   \item Observe that \begin{align*}|g(x)-g(y)|&=\left|\int_y^xf(t)dt\right|\\&\leq\int_y^x|f(t)|dt\\&\leq\left(\int_y^x|f(t)|^pdt\right)^{1/p}|x-y|^{1-\frac{1}{p}}\\&\leq\|f\|_{L^p(\R^n)}|x-y|^{\alpha}\end{align*}
  \end{enumerate}
\end{Proof}
\end{enumerate}
\end{document}